\documentclass[10pt]{article}
\usepackage[utf8]{inputenc}
\usepackage{amsmath}
\usepackage{amsthm}
\usepackage{amssymb}
\usepackage[a4paper, margin = .5in]{geometry}
\usepackage{float}
\usepackage{graphicx}
\usepackage{pdflscape}
\usepackage{booktabs}
\usepackage{threeparttable}
\usepackage{grffile}
\usepackage{adjustbox}
\usepackage{mathpazo}

\usepackage[normalem]{ulem}


\usepackage{hyperref}
\hypersetup{
    colorlinks,
    citecolor= blue,
    filecolor= black,
    linkcolor= black,
    urlcolor= black
}

\newcommand{\tab}{\quad}



\title{Results for PEACE Follow Up}
\date{\today}

\begin{document}
\maketitle 
\tableofcontents
\listoftables
\newpage
\clearpage

\section{Explanation}

\subsection*{Reviewers are concerned with who in particular feels insecure about their property. Which groups feel insecure?} 

\par Table \ref{summary_stats_demo} shows means of all key outcomes across various demographic dummy variables. Security rights variables are in the bottom, in panel E. I have not yet done a regression analysis for these variables yet, however I believe the current table is still informative. Please let me know if you want to see this exact table, but with regression estimates instead. 

\subsection*{Reviewers are concerned about us dropping communities between Endline 1 and Endline 2. What was the process used to select which communities to drop?}

\begin{enumerate}
    \item Enumerators gave use a list of 68 communities (some of whom were treated) that they deemed difficult to reach.
    \item We ran a probit regression to predict, for each community, the probability of assigment to treatment. 
    \item Of the "difficult to reach" communities, of which 41 were eligible to be dropped, we dropped 20 and made a weight  for the rest that is 43/20
    \item Of the communities in the bottom 35\% of predicted probability of assignment to treatment, of which 41 were eligible to be dropped, we dropped 20 and make a weight for the rest that is 41/20. 
\end{enumerate}

What defines being eligible to be dropped? 

\begin{itemize}
    \item Being a control community
    \item Not being a control area in a in a larger community where another area in that community is treated (this is just one community). 
\end{itemize}

\subsection*{Do the weights affect our Endline 1 results at all?}

Table \ref{conflict_adj_p_dropped} answers this question. It shows two regressions. First, a regression with the Endline communities with Endline 2 weights. This means that the communities that were dropped are not in the regression, and communities that \textit{could have been dropped}, but weren't, are weighted more heavily. As you can see, all of our main effects hold. 

Second, it shows the effect of being a dropped community on all outcomes. There are no weights in this regression. As you can see, being a dropped is never a significant predictor of violence. 
\begin{itemize}
    \item \textit{Note: I do not yet have a pretty table showing correlates of being dropped, but can make one if we still think it is necessary.} 
    \item \textit{Additional note: An earlier version of Table \ref{conflict_adj_p_dropped} where we had only the sample of non-dropped communities, but without additional weighting. I toyed with the idea of putting that regression in Table \ref{conflict_adj_p_dropped} as well, but decided there was enough information in the table already. Results are the exact same.}
\end{itemize}

\subsection*{Reviewers are concerned about how we handle multiple hypotheses. What are we doing to respond?}
In all of our ITT tables, we have corrected for multiple outcomes using the Westfall-Young method and the Sidak-Holmes method. In particular, we are using the \texttt{wyoung} package from \texttt{ssc} in Stata. The Westfall-Young approach involves constructing simulated p-values from bootstrapping (sampling with replacement). We came into this process under the impression that the Westfall-Young method would be less restrictive than the Holmes method, but this has not been the case. In all likelihood this is because out outcome measures are noisy. 

Tables \ref{land_conflict_paper_year} and \ref{all_conflict_paper_year} have a unique set of corrections, where we avoid correcting for the sub-components of indices. Table \ref{conflict_adj_p_year} shows adjusted results for land and general conflict in the same family. All other tables just have all outcomes in the table placed into the same family. 





\section{Paper Tables}
\begin{table}[H]
\caption{ Program impacts on number, length, severity, and resolution of land disputes}
\label{land_conflict_paper_year}
\begin{center}
\begin{adjustbox}{max width = \textwidth}
\begin{tabular}{lcccccccccccccc}
\hline \noalign{\smallskip} & \multicolumn{7}{c}{\uline{\hfill Endline 1 \hfill}} & \multicolumn{7}{c}{\uline{\hfill Endline 2 \hfill}}\\
 &  &  &  & ITT / &  &  &  &  &  &  & ITT / &  &  & \\
 & Control &  &  & control & Est. & WY Adj. & Holms Adj & Control &  &  & control & Est. & WY Adj. & Holms Adj\\
Dependent Variable & mean & N & ITT & mean (\%) & p-val & p-val & p-val & mean & N & ITT & mean (\%) & p-val & p-val & p-val\\
 & (1) & (2) & (3) & (4) & (5) & (6) & (7) & (8) & (9) & (10) & (11) & (12) & (13) & (14)\\
\noalign{\smallskip}\hline \noalign{\smallskip}\textbf{Panel A: Ouctomes for all residents} &  &  &  &  &  &  &  &  &  &  &  &  &  & \\
 &  &  &  &  &  &  &  &  &  &  &  &  &  & \\
Any serious land dispute & 0.221 & 5,435 & 0.003 & 1.2 & 0.867 & 1.000\textsuperscript{a} & 0.867 & 0.087 & 4,011 & 0.008 & 8.8 & 0.473 & 1.000\textsuperscript{b} & 0.854\\
 &  &  & [0.016] &  &  &  &  &  &  & [0.011] &  &  &  & \\
Any unresolved land dispute & 0.070 & 5,435 & -0.020 & -28.0 & 0.015 & 0.000\textsuperscript{a} & 0.057 & 0.024 & 4,011 & 0.002 & 6.4 & 0.744 & 1.000\textsuperscript{b} & 0.854\\
 &  &  & [0.008]** &  &  &  &  &  &  & [0.005] &  &  &  & \\
Any threats, property damage, or violence & 0.122 & 5,435 & -0.010 & -8.1 & 0.397 & 0.500\textsuperscript{a} & 0.781 & 0.041 & 4,011 & -0.012 & -29.3 & 0.039 & 0.500\textsuperscript{b} & 0.182\\
 &  &  & [0.012] &  &  &  &  &  &  & [0.006]** &  &  &  & \\
\quad Property damage or violence in land dispute & 0.091 & 5,435 & -0.012 & -13.2 & 0.183 &  &  & 0.021 & 4,011 & -0.007 & -31.2 & 0.117 &  & \\
 &  &  & [0.009] &  &  &  &  &  &  & [0.004] &  &  &  & \\
\tab Threats & 0.114 & 5,435 & -0.006 & -5.3 & 0.596 &  &  & 0.035 & 4,011 & -0.010 & -28.9 & 0.069 &  & \\
 &  &  & [0.011] &  &  &  &  &  &  & [0.006]* &  &  &  & \\
\tab Property damage & 0.041 & 5,435 & -0.013 & -32.4 & 0.029 &  &  & 0.010 & 4,011 & -0.005 &  & 0.072 &  & \\
 &  &  & [0.006]** &  &  &  &  &  &  & [0.003]* &  &  &  & \\
\tab Violence & 0.077 & 5,435 & -0.007 & -8.7 & 0.416 &  &  & 0.017 & 4,011 & -0.004 & -21.4 & 0.361 &  & \\
 &  &  & [0.008] &  &  &  &  &  &  & [0.004] &  &  &  & \\
\textbf{Panel B: Conditional on a land dispute} &  &  &  &  &  &  &  &  &  &  &  &  &  & \\
 &  &  &  &  &  &  &  &  &  &  &  &  &  & \\
Length of maximum land conflict &  &  &  &  &  &  &  & 13.247 & 353 & 3.642 & 27.5 & 0.209 & 0.500\textsuperscript{b} & 0.609\\
 &  &  &  &  &  &  &  &  &  & [2.889] &  &  &  & \\
Resolved land dispute & 0.684 & 1,212 & 0.072 & 10.5 & 0.009 & 0.000\textsuperscript{a} & 0.045 & 0.668 & 353 & -0.024 & -3.6 & 0.604 & 1.000\textsuperscript{b} & 0.854\\
 &  &  & [0.027]*** &  &  &  &  &  &  & [0.046] &  &  &  & \\
Any threats, property damage, or violence \phantom{} & 0.554 & 1,212 & -0.024 & -4.3 & 0.496 & 0.500\textsuperscript{a} & 0.781 & 0.476 & 353 & -0.192 & -40.4 & 0.000 & 0.000\textsuperscript{b} & 0.000\\
 &  &  & [0.035] &  &  &  &  &  &  & [0.047]*** &  &  &  & \\
\quad Property damage or violence \tab & 0.411 & 1,212 & -0.037 & -9.1 & 0.213 &  &  & 0.243 & 353 & -0.090 & -37.2 & 0.035 &  & \\
 &  &  & [0.030] &  &  &  &  &  &  & [0.042]** &  &  &  & \\
\tab Threats \phantom{} & 0.515 & 1,212 & -0.013 & -2.5 & 0.713 &  &  & 0.408 & 353 & -0.157 & -38.6 & 0.001 &  & \\
 &  &  & [0.035] &  &  &  &  &  &  & [0.048]*** &  &  &  & \\
\tab Property damage \phantom{} & 0.186 & 1,212 & -0.051 & -27.4 & 0.041 &  &  & 0.114 & 353 & -0.067 & -58.5 & 0.016 &  & \\
 &  &  & [0.025]** &  &  &  &  &  &  & [0.027]** &  &  &  & \\
\tab Violence \phantom{} & 0.349 & 1,212 & -0.022 & -6.3 & 0.442 &  &  & 0.202 & 353 & -0.056 & -28.0 & 0.186 &  & \\
 &  &  & [0.028] &  &  &  &  &  &  & [0.042] &  &  &  & \\
\tab Witchcraft \phantom{} &  &  &  &  &  &  &  & 0.065 & 353 & 0.035 & 54.3 & 0.182 &  & \\
 &  &  &  &  &  &  &  &  &  & [0.026] &  &  &  & \\
\noalign{\smallskip}\hline\end{tabular}

\end{adjustbox}
\end{center}
\end{table}

\begin{table}[H]
\caption{Program impacts on number, length, severity, and resolution of all dispute types}
\begin{center}
\begin{adjustbox}{max width = \textwidth}
\label{all_conflict_paper_year}
\begin{tabular}{lcccccccccccccc}
\hline \noalign{\smallskip} & \multicolumn{7}{c}{\uline{\hfill Endline 1 \hfill}} & \multicolumn{7}{c}{\uline{\hfill Endline 2 \hfill}}\\
 &  &  &  & ITT / &  &  &  &  & ITT / &  &  &  &  & \\
 & Control &  &  & control & Est. & WY Adj. & Holms Adj & Control &  &  & control & Est. & WY Adj. & Holms Adj\\
Dependent Variable & mean & N & ITT & mean (\%) & p-val & p-val & p-val & mean & N & ITT & mean (\%) & p-val & p-val & p-val\\
 & (1) & (2) & (3) & (4) & (5) & (6) & (7) & (8) & (9) & (10) & (11) & (12) & (13) & (14)\\
\noalign{\smallskip}\hline \noalign{\smallskip}\textbf{Panel A: Ouctomes for all residents} &  &  &  &  &  &  &  &  &  &  &  &  &  & \\
 &  &  &  &  &  &  &  &  &  &  &  &  &  & \\
Any serious dispute & 0.299 & 5,435 & 0.022 & 7.2 & 0.220 & 0.406\textsuperscript{a} & 0.391 & 0.306 & 4,011 & 0.012 & 4.1 & 0.456 & 0.717\textsuperscript{b} & 0.703\\
 &  &  & [0.018] &  &  &  &  &  &  & [0.017] &  &  &  & \\
Any unresolved dispute & 0.118 & 5,435 & -0.013 & -11.0 & 0.250 & 0.406\textsuperscript{a} & 0.391 & 0.064 & 4,011 & -0.004 & -6.6 & 0.623 & 0.717\textsuperscript{b} & 0.703\\
 &  &  & [0.011] &  &  &  &  &  &  & [0.009] &  &  &  & \\
\phantom{} Any threats, property damage, or violence &  &  &  &  &  &  &  & 0.101 & 4,011 & -0.015 & -15.2 & 0.111 & 0.357\textsuperscript{b} & 0.295\\
 &  &  &  &  &  &  &  &  &  & [0.010] &  &  &  & \\
\textbf{Panel B: Conditional on dispute} &  &  &  &  &  &  &  &  &  &  &  &  &  & \\
 &  &  &  &  &  &  &  &  &  &  &  &  &  & \\
Resolved dispute & 0.676 & 1,670 & 0.051 & 7.5 & 0.033 & 0.084\textsuperscript{c} & 0.065 & 0.767 & 1,227 & -0.019 & -2.5 & 0.467 & 0.726\textsuperscript{d} & 0.715\\
 &  &  & [0.024]** &  &  &  &  &  &  & [0.026] &  &  &  & \\
\quad Resolved via informal mechanism & 0.247 & 1,671 & 0.018 & 7.4 & 0.470 & 0.494\textsuperscript{c} & 0.470 & 0.409 & 1,227 & -0.019 & -4.6 & 0.487 & 0.726\textsuperscript{d} & 0.715\\
 &  &  & [0.025] &  &  &  &  &  &  & [0.027] &  &  &  & \\
\phantom{} Any threats, property damage, or violence \phantom{} &  &  &  &  &  &  &  & 0.331 & 1,227 & -0.069 & -20.7 & 0.009 & 0.063\textsuperscript{d} & 0.026\\
 &  &  &  &  &  &  &  &  &  & [0.026]*** &  &  &  & \\
\quad Property damage or violence \phantom{} &  &  &  &  &  &  &  & 0.190 & 1,227 & -0.027 & -14.1 & 0.222 &  & \\
 &  &  &  &  &  &  &  &  &  & [0.022] &  &  &  & \\
\quad Threats \phantom{} &  &  &  &  &  &  &  & 0.274 & 1,227 & -0.072 & -26.2 & 0.003 &  & \\
 &  &  &  &  &  &  &  &  &  & [0.024]*** &  &  &  & \\
\quad Property damage \phantom{} &  &  &  &  &  &  &  & 0.048 & 1,227 & -0.019 & -40.0 & 0.085 &  & \\
 &  &  &  &  &  &  &  &  &  & [0.011]* &  &  &  & \\
\quad Violence \phantom{} &  &  &  &  &  &  &  & 0.173 & 1,227 & -0.018 & -10.2 & 0.411 &  & \\
 &  &  &  &  &  &  &  &  &  & [0.021] &  &  &  & \\
\noalign{\smallskip}\hline\end{tabular}

\end{adjustbox}
\end{center}
\end{table}

\begin{table}[H]
\caption{Effect on land security and investment 3-year endline}
\begin{center}
\begin{adjustbox}{max width = \textwidth}
\begin{tabular}{lcccccc}
\hline \noalign{\smallskip} &  &  &  &  &  & \\
 &  & Control &  & Est. & Wy adj. & Holms adj.\\
Dependent Variable & N & mean & ITT & p-val & p-val & p-val\\
 & (1) & (2) & (3) & (4) & (5) & (6)\\
\noalign{\smallskip}\hline \noalign{\smallskip}Security rights index, z-score & 4,011 & 0.045 & -0.085 & 0.024 & 0.400 & 0.071\\
 &  &  & [0.037]** &  &  & \\
Improvement index, z-score & 4,011 & 0.023 & -0.066 & 0.079 & 0.600 & 0.151\\
 &  &  & [0.037]* &  &  & \\
Index of Fallow Land for farm & 3,666 & 0.003 & -0.004 & 0.926 & 0.800 & 0.926\\
 &  &  & [0.044] &  &  & \\
Size of farm & 3,598 & 37.481 & 2.773 & 0.009 & 0.200 & 0.036\\
 &  &  & [1.051]*** &  &  & \\
\noalign{\smallskip}\hline\end{tabular}

\end{adjustbox}
\end{center}
\end{table}

\begin{table}[H]
\caption{Heterogeneity in land security and investment, 3-year endline}
\begin{center}
\begin{adjustbox}{max width = \textwidth}
\begin{tabular}{lccccccccc}
\hline \noalign{\smallskip} & \multicolumn{3}{c}{\uline{\hfill Political Connectedness \hfill}} & \multicolumn{3}{c}{\uline{\hfill Market Tenure \hfill}} & \multicolumn{3}{c}{\uline{\hfill Owns own land \hfill}}\\
 &  & Coeff. on &  &  & Coeff. on &  &  & Coeff. on & \\
 &  & treatment- & Net &  & treatment- & Net &  & treatment- & Net\\
 & Coeff. on & covariate & effect & Coeff. on & covariate & effect & Coeff. on & covariate & effect\\
Dependent variable & treatment & interaction & (sum) & treatment & interaction & (sum) & treatment & interaction & (sum)\\
\noalign{\smallskip}\hline \noalign{\smallskip}Security index through rights & -0.126 & 0.140 & 0.014 & -0.129 & 0.083 & -0.046 & -0.087 & 0.058 & -0.030\\
 & \begin{footnotesize}[0.037]***\end{footnotesize} & \begin{footnotesize}[0.058]**\end{footnotesize} & \begin{footnotesize}[0.047]\end{footnotesize} & \begin{footnotesize}[0.063]**\end{footnotesize} & \begin{footnotesize}[0.067]\end{footnotesize} & \begin{footnotesize}[0.031]\end{footnotesize} & \begin{footnotesize}[0.031]***\end{footnotesize} & \begin{footnotesize}[0.076]\end{footnotesize} & \begin{footnotesize}[0.072]\end{footnotesize}\\
\noalign{\smallskip}Property investment index, z-score & -0.046 & -0.022 & -0.068 & -0.013 & -0.044 & -0.056 & -0.049 & -0.017 & -0.067\\
 & \begin{footnotesize}[0.031]\end{footnotesize} & \begin{footnotesize}[0.054]\end{footnotesize} & \begin{footnotesize}[0.046]\end{footnotesize} & \begin{footnotesize}[0.050]\end{footnotesize} & \begin{footnotesize}[0.054]\end{footnotesize} & \begin{footnotesize}[0.029]**\end{footnotesize} & \begin{footnotesize}[0.028]*\end{footnotesize} & \begin{footnotesize}[0.064]\end{footnotesize} & \begin{footnotesize}[0.060]\end{footnotesize}\\
\noalign{\smallskip}Size of land (house and farm) & 2.235 & 1.496 & 3.731 & 2.689 & -0.152 & 2.537 & 3.247 & -3.106 & 0.141\\
 & \begin{footnotesize}[1.137]**\end{footnotesize} & \begin{footnotesize}[1.956]\end{footnotesize} & \begin{footnotesize}[1.653]**\end{footnotesize} & \begin{footnotesize}[2.082]\end{footnotesize} & \begin{footnotesize}[2.303]\end{footnotesize} & \begin{footnotesize}[1.057]**\end{footnotesize} & \begin{footnotesize}[1.034]***\end{footnotesize} & \begin{footnotesize}[2.513]\end{footnotesize} & \begin{footnotesize}[2.327]\end{footnotesize}\\
\noalign{\smallskip}\hline\end{tabular}

\end{adjustbox}
\end{center}
\end{table}

\begin{table}[H]
\caption{Effect on norms, attitudes and skills, 3-year endline}
\begin{center}
\begin{adjustbox}{max width = \textwidth}
\begin{tabular}{lcccccc}
\hline \noalign{\smallskip} &  &  &  &  &  & \\
 &  & Control &  & Est. & Wy adj. & Holms adj.\\
Dependent Variable & N & mean & ITT & p-val & p-val & p-val\\
 & (1) & (2) & (3) & (4) & (5) & (6)\\
\noalign{\smallskip}\hline \noalign{\smallskip}Index of all norms combined & 4,011 & -0.026 & 0.028 & 0.469 & 0.800 & 0.895\\
 &  &  & [0.038] &  &  & \\
\quad Bias index & 4,011 & -0.009 & -0.002 & 0.962 & 1.000 & 0.997\\
 &  &  & [0.046] &  &  & \\
<<<<<<< HEAD
\quad Defection index & 4,011 & -0.043 & 0.045 & 0.274 & 0.600 & 0.853\\
=======
\quad Defection index & 4,011 & -0.043 & 0.045 & 0.274 & 0.800 & 0.853\\
>>>>>>> 0d75059017c53f4b3b73a7a8b0fa772b25b65cb1
 &  &  & [0.041] &  &  & \\
\quad Empathy index & 4,010 & 0.002 & 0.030 & 0.363 & 0.800 & 0.895\\
 &  &  & [0.033] &  &  & \\
\quad Forum choice index & 4,011 & -0.028 & 0.031 & 0.400 & 0.800 & 0.895\\
 &  &  & [0.037] &  &  & \\
<<<<<<< HEAD
\quad Managing emotions index & 4,011 & -0.031 & 0.067 & 0.032 & 0.200 & 0.228\\
 &  &  & [0.031]** &  &  & \\
\quad Mediation index & 4,011 & 0.003 & -0.062 & 0.094 & 0.400 & 0.499\\
=======
\quad Managing emotions index & 4,011 & -0.031 & 0.067 & 0.032 & 0.400 & 0.228\\
 &  &  & [0.031]** &  &  & \\
\quad Mediation index & 4,011 & 0.003 & -0.062 & 0.094 & 0.800 & 0.499\\
>>>>>>> 0d75059017c53f4b3b73a7a8b0fa772b25b65cb1
 &  &  & [0.037]* &  &  & \\
\quad Negotiation index & 4,011 & 0.002 & 0.002 & 0.945 & 1.000 & 0.997\\
 &  &  & [0.027] &  &  & \\
\noalign{\smallskip}\hline\end{tabular}

\end{adjustbox}
\end{center}
\end{table}

\begin{table}[H]
\caption{Effect on community-level disputes}
\begin{center}
\begin{adjustbox}{max width = \textwidth}
\begin{tabular}{lcccccccccccccc}
\hline \noalign{\smallskip} & \multicolumn{7}{c}{\uline{\hfill 1-year endline \hfill}} & \multicolumn{7}{c}{\uline{\hfill 3-year endline \hfill}}\\
 &  &  &  & ITT / &  &  &  &  &  &  & ITT / &  &  & \\
 & Control &  &  & control & Est. & WY Adj. & Holms Adj & Control &  &  & control & Est. & WY Adj. & Holms Adj\\
Dependent Variable & mean & N & ITT & mean (\%) & p-val & p-val & p-val & mean & N & ITT & mean (\%) & p-val & p-val & p-val\\
 & (1) & (2) & (3) & (4) & (5) & (6) & (7) & (8) & (9) & (10) & (11) & (12) & (13) & (14)\\
\noalign{\smallskip}\hline \noalign{\smallskip}Any Violence & 0.442 & 940 & 0.085 & 19.3 & 0.021 & 0.500\textsuperscript{a} & 0.193 & 0.622 & 971 & -0.057 & -9.1 & 0.319 & 1.000\textsuperscript{b} & 0.735\\
 &  &  & [0.037]** &  &  &  &  &  &  & [0.057] &  &  &  & \\
Level of community violence & 0.790 & 940 & 0.152 & 19.2 & 0.070 & 1.000\textsuperscript{a} & 0.478 & 0.984 & 971 & -0.131 & -13.3 & 0.171 & 1.000\textsuperscript{b} & 0.677\\
 &  &  & [0.083]* &  &  &  &  &  &  & [0.096] &  &  &  & \\
\quad Intertribal violence & 0.028 & 940 & 0.008 & 30.0 & 0.512 & 1.000\textsuperscript{a} & 0.884 & 0.021 & 971 & -0.016 & -77.8 & 0.039 & 0.500\textsuperscript{b} & 0.331\\
 &  &  & [0.013] &  &  &  &  &  &  & [0.008]** &  &  &  & \\
\quad Violent strike or protest & 0.061 & 940 & -0.004 & -7.1 & 0.782 & 1.000\textsuperscript{a} & 0.952 & 0.002 & 971 & 0.015 & 912.4 & 0.104 & 0.500\textsuperscript{b} & 0.585\\
 &  &  & [0.016] &  &  &  &  &  &  & [0.009] &  &  &  & \\
\quad Youth-elder dispute & 0.110 & 940 & 0.044 & 40.2 & 0.124 & 1.000\textsuperscript{a} & 0.603 & 0.103 & 967 & 0.003 & 3.4 & 0.880 & 1.000\textsuperscript{b} & 0.880\\
 &  &  & [0.029] &  &  &  &  &  &  & [0.023] &  &  &  & \\
\quad Peaceful strike or protest & 0.100 & 940 & 0.037 & 37.1 & 0.144 & 1.000\textsuperscript{a} & 0.606 & 0.059 & 971 & 0.010 & 16.0 & 0.613 & 1.000\textsuperscript{b} & 0.850\\
 &  &  & [0.025] &  &  &  &  &  &  & [0.019] &  &  &  & \\
\quad Interfamily land disputes & 0.274 & 940 & 0.030 & 11.1 & 0.408 & 1.000\textsuperscript{a} & 0.878 & 0.548 & 971 & -0.071 & -12.9 & 0.282 & 1.000\textsuperscript{b} & 0.735\\
 &  &  & [0.037] &  &  &  &  &  &  & [0.066] &  &  &  & \\
\quad Conflicts with other towns & 0.154 & 940 & -0.005 & -3.5 & 0.870 & 1.000\textsuperscript{a} & 0.952 & 0.171 & 970 & -0.038 & -22.2 & 0.194 & 1.000\textsuperscript{b} & 0.677\\
 &  &  & [0.033] &  &  &  &  &  &  & [0.029] &  &  &  & \\
\quad Witch hunts & 0.015 & 940 & 0.023 & 153.3 & 0.087 & 1.000\textsuperscript{a} & 0.517 & 0.011 & 971 & -0.008 & -71.8 & 0.085 & 0.500\textsuperscript{b} & 0.551\\
 &  &  & [0.013]* &  &  &  &  &  &  & [0.005]* &  &  &  & \\
\quad Trial by ordeal & 0.048 & 940 & 0.019 & 39.4 & 0.298 & 1.000\textsuperscript{a} & 0.830 & 0.070 & 971 & -0.027 & -39.1 & 0.120 & 1.000\textsuperscript{b} & 0.592\\
 &  &  & [0.018] &  &  &  &  &  &  & [0.018] &  &  &  & \\
\noalign{\smallskip}\hline\end{tabular}

\end{adjustbox}
\end{center}
\end{table}

\begin{table}[H]
\caption{Estimated aggregate effects of the program on violent disputes among the 30,000 households in treatment communities}
\begin{center}
\begin{adjustbox}{max width = \textwidth}
\begin{tabular}{lccccc}
\hline \noalign{\smallskip} & \multicolumn{2}{c}{\uline{\hfill ATE \hfill}} & \multicolumn{2}{c}{\uline{\hfill Net effect \hfill}} & Net Effect imputed\\
Dependent Variable & 1-year endline & 3-year endline & 1-year endline & 3-year endline & for missing year\\
 & (1) & (2) & (3) & (4) & (5)\\
\noalign{\smallskip}\hline \noalign{\smallskip}Any threats, property damage, or violence & -0.010 & -0.012 & -303 & -325 & -942\\
\tab Threats & -0.006 & -0.010 & -183 & -275 & -687\\
\quad Property damage or violence in land dispute & -0.012 & -0.007 & -365 & -177 & -813\\
\tab Property damage & -0.013 & -0.005 & -406 & -139 & -817\\
\tab Violence & -0.007 & -0.004 & -206 & -101 & -459\\
\quad Property damage + violence (land) & -0.020 & -0.009 & -611 & -240 & -1,276\\
\noalign{\smallskip}\hline\end{tabular}

\end{adjustbox}
\end{center}
\end{table}

\section{Appendix Tables}
\clearpage
\setcounter{table}{0}   
\renewcommand{\thetable}{A.\arabic{table}}

\begin{table}[H]
\caption{2008 baseline summary statistics and test of randomization balance}
\begin{center}
\begin{adjustbox}{max width = \textwidth}
\end{adjustbox}
\end{center}
\end{table}


\begin{table}[H]
\caption{Effects of intense treatment}
\begin{center}
\begin{adjustbox}{max width = \textwidth}
\begin{tabular}{lcccccc}
\hline \noalign{\smallskip} & \multicolumn{3}{c}{\uline{\hfill Endline 1 \hfill}} & \multicolumn{3}{c}{\uline{\hfill Endline 2 \hfill}}\\
 & Normal & Intensive &  & Normal & Intensive & \\
 & treatment & treatment & Sum & treatment & treatment & Sum\\
\noalign{\smallskip}\hline \noalign{\smallskip}Resident: If assigned: Attended peace program & 0.251 & 0.086 & 0.336 &  &  & \\
 & \begin{footnotesize}[0.023]***\end{footnotesize} & \begin{footnotesize}[0.045]*\end{footnotesize} & \begin{footnotesize}[0.040]***\end{footnotesize} & \begin{footnotesize}\end{footnotesize} & \begin{footnotesize}\end{footnotesize} & \begin{footnotesize}\end{footnotesize}\\
\noalign{\smallskip}Any serious land dispute & 0.010 & -0.030 & -0.020 & 0.007 & -0.013 & -0.006\\
 & \begin{footnotesize}[0.018]\end{footnotesize} & \begin{footnotesize}[0.030]\end{footnotesize} & \begin{footnotesize}[0.028]\end{footnotesize} & \begin{footnotesize}[0.012]\end{footnotesize} & \begin{footnotesize}[0.016]\end{footnotesize} & \begin{footnotesize}[0.014]\end{footnotesize}\\
\noalign{\smallskip}Any threats, property damage, or violence & -0.008 & -0.008 & -0.016 & -0.013 & -0.006 & -0.019\\
 & \begin{footnotesize}[0.013]\end{footnotesize} & \begin{footnotesize}[0.021]\end{footnotesize} & \begin{footnotesize}[0.019]\end{footnotesize} & \begin{footnotesize}[0.007]*\end{footnotesize} & \begin{footnotesize}[0.009]\end{footnotesize} & \begin{footnotesize}[0.007]**\end{footnotesize}\\
\noalign{\smallskip}index of security through rights, for respondent &  &  &  & -0.083 & 0.008 & -0.075\\
 & \begin{footnotesize}\end{footnotesize} & \begin{footnotesize}\end{footnotesize} & \begin{footnotesize}\end{footnotesize} & \begin{footnotesize}[0.044]*\end{footnotesize} & \begin{footnotesize}[0.064]\end{footnotesize} & \begin{footnotesize}[0.054]\end{footnotesize}\\
\noalign{\smallskip}Index of improvement, for respondent &  &  &  & -0.066 & 0.027 & -0.039\\
 & \begin{footnotesize}\end{footnotesize} & \begin{footnotesize}\end{footnotesize} & \begin{footnotesize}\end{footnotesize} & \begin{footnotesize}[0.039]*\end{footnotesize} & \begin{footnotesize}[0.061]\end{footnotesize} & \begin{footnotesize}[0.056]\end{footnotesize}\\
\noalign{\smallskip}\hline\end{tabular}

\end{adjustbox}
\end{center}
\end{table}

\begin{landscape}
\begin{table}[H]
\caption{Heterogeneity by demographics and peace background on conflict}
\begin{center}
\begin{adjustbox}{max width = 1.5\textheight}
\begin{tabular}{lcccccccccccccccccc}
\hline \noalign{\smallskip} & \multicolumn{3}{c}{\uline{\hfill Female \hfill}} & \multicolumn{3}{c}{\uline{\hfill 20-40 years old \hfill}} & \multicolumn{3}{c}{\uline{\hfill Wealth index \hfill}} & \multicolumn{3}{c}{\uline{\hfill Any ethnic minority \hfill}} & \multicolumn{3}{c}{\uline{\hfill \% town peace education at baseline \hfill}} & \multicolumn{3}{c}{\uline{\hfill \% town peace group at baseline \hfill}}\\
 & Treatment & Interaction & Sum & Treatment & Interaction & Sum & Treatment & Interaction & Sum & Treatment & Interaction & Sum & Treatment & Interaction & Sum & Treatment & Interaction & Sum\\
\noalign{\smallskip}\hline \noalign{\smallskip}Any serious land dispute & -0.002 & 0.018 & 0.016 & 0.014 & -0.012 & 0.002 & 0.008 & -0.016 & -0.008 & 0.009 & -0.013 & -0.004 & 0.030 & -0.079 & -0.049 & 0.014 & -0.016 & -0.002\\
 & \begin{footnotesize}[0.015]\end{footnotesize} & \begin{footnotesize}[0.018]\end{footnotesize} & \begin{footnotesize}[0.013]\end{footnotesize} & \begin{footnotesize}[0.014]\end{footnotesize} & \begin{footnotesize}[0.017]\end{footnotesize} & \begin{footnotesize}[0.013]\end{footnotesize} & \begin{footnotesize}[0.011]\end{footnotesize} & \begin{footnotesize}[0.017]\end{footnotesize} & \begin{footnotesize}[0.021]\end{footnotesize} & \begin{footnotesize}[0.011]\end{footnotesize} & \begin{footnotesize}[0.026]\end{footnotesize} & \begin{footnotesize}[0.024]\end{footnotesize} & \begin{footnotesize}[0.017]*\end{footnotesize} & \begin{footnotesize}[0.051]\end{footnotesize} & \begin{footnotesize}[0.039]\end{footnotesize} & \begin{footnotesize}[0.021]\end{footnotesize} & \begin{footnotesize}[0.048]\end{footnotesize} & \begin{footnotesize}[0.032]\end{footnotesize}\\
\noalign{\smallskip}Any unresolved land dispute & 0.005 & -0.008 & -0.002 & 0.005 & -0.006 & -0.001 & 0.002 & -0.011 & -0.010 & -0.001 & 0.025 & 0.023 & 0.006 & -0.019 & -0.013 & 0.009 & -0.019 & -0.010\\
 & \begin{footnotesize}[0.008]\end{footnotesize} & \begin{footnotesize}[0.010]\end{footnotesize} & \begin{footnotesize}[0.006]\end{footnotesize} & \begin{footnotesize}[0.006]\end{footnotesize} & \begin{footnotesize}[0.009]\end{footnotesize} & \begin{footnotesize}[0.007]\end{footnotesize} & \begin{footnotesize}[0.005]\end{footnotesize} & \begin{footnotesize}[0.008]\end{footnotesize} & \begin{footnotesize}[0.009]\end{footnotesize} & \begin{footnotesize}[0.005]\end{footnotesize} & \begin{footnotesize}[0.013]*\end{footnotesize} & \begin{footnotesize}[0.012]*\end{footnotesize} & \begin{footnotesize}[0.007]\end{footnotesize} & \begin{footnotesize}[0.024]\end{footnotesize} & \begin{footnotesize}[0.019]\end{footnotesize} & \begin{footnotesize}[0.009]\end{footnotesize} & \begin{footnotesize}[0.021]\end{footnotesize} & \begin{footnotesize}[0.014]\end{footnotesize}\\
\noalign{\smallskip}\textbf{Conditional on a land dispute} &  &  &  &  &  &  &  &  &  &  &  &  &  &  &  &  &  & \\
 & \begin{footnotesize}\end{footnotesize} & \begin{footnotesize}\end{footnotesize} & \begin{footnotesize}\end{footnotesize} & \begin{footnotesize}\end{footnotesize} & \begin{footnotesize}\end{footnotesize} & \begin{footnotesize}\end{footnotesize} & \begin{footnotesize}\end{footnotesize} & \begin{footnotesize}\end{footnotesize} & \begin{footnotesize}\end{footnotesize} & \begin{footnotesize}\end{footnotesize} & \begin{footnotesize}\end{footnotesize} & \begin{footnotesize}\end{footnotesize} & \begin{footnotesize}\end{footnotesize} & \begin{footnotesize}\end{footnotesize} & \begin{footnotesize}\end{footnotesize} & \begin{footnotesize}\end{footnotesize} & \begin{footnotesize}\end{footnotesize} & \begin{footnotesize}\end{footnotesize}\\
\noalign{\smallskip}Any threats, property damage, or violence \phantom{} & -0.229 & 0.079 & -0.150 & -0.192 & -0.001 & -0.193 & -0.190 & -0.066 & -0.256 & -0.185 & -0.068 & -0.253 & -0.206 & 0.086 & -0.120 & -0.251 & 0.152 & -0.099\\
 & \begin{footnotesize}[0.066]***\end{footnotesize} & \begin{footnotesize}[0.104]\end{footnotesize} & \begin{footnotesize}[0.074]**\end{footnotesize} & \begin{footnotesize}[0.071]***\end{footnotesize} & \begin{footnotesize}[0.107]\end{footnotesize} & \begin{footnotesize}[0.071]***\end{footnotesize} & \begin{footnotesize}[0.047]***\end{footnotesize} & \begin{footnotesize}[0.079]\end{footnotesize} & \begin{footnotesize}[0.089]***\end{footnotesize} & \begin{footnotesize}[0.051]***\end{footnotesize} & \begin{footnotesize}[0.143]\end{footnotesize} & \begin{footnotesize}[0.130]*\end{footnotesize} & \begin{footnotesize}[0.106]*\end{footnotesize} & \begin{footnotesize}[0.327]\end{footnotesize} & \begin{footnotesize}[0.238]\end{footnotesize} & \begin{footnotesize}[0.124]**\end{footnotesize} & \begin{footnotesize}[0.273]\end{footnotesize} & \begin{footnotesize}[0.165]\end{footnotesize}\\
\noalign{\smallskip}Resolved land dispute & -0.062 & 0.081 & 0.019 & -0.133 & 0.201 & 0.068 & -0.021 & -0.085 & -0.106 & -0.008 & -0.153 & -0.161 & -0.049 & 0.083 & 0.033 & 0.048 & -0.185 & -0.137\\
 & \begin{footnotesize}[0.067]\end{footnotesize} & \begin{footnotesize}[0.102]\end{footnotesize} & \begin{footnotesize}[0.070]\end{footnotesize} & \begin{footnotesize}[0.075]*\end{footnotesize} & \begin{footnotesize}[0.109]*\end{footnotesize} & \begin{footnotesize}[0.069]\end{footnotesize} & \begin{footnotesize}[0.046]\end{footnotesize} & \begin{footnotesize}[0.074]\end{footnotesize} & \begin{footnotesize}[0.089]\end{footnotesize} & \begin{footnotesize}[0.050]\end{footnotesize} & \begin{footnotesize}[0.138]\end{footnotesize} & \begin{footnotesize}[0.128]\end{footnotesize} & \begin{footnotesize}[0.097]\end{footnotesize} & \begin{footnotesize}[0.270]\end{footnotesize} & \begin{footnotesize}[0.191]\end{footnotesize} & \begin{footnotesize}[0.109]\end{footnotesize} & \begin{footnotesize}[0.248]\end{footnotesize} & \begin{footnotesize}[0.155]\end{footnotesize}\\
\noalign{\smallskip}\hline\end{tabular}

\end{adjustbox}
\end{center}
\end{table}

\begin{table}[H]
\caption{Heterogeneity by demographics and peace background on security and investment}
\begin{center}
\begin{adjustbox}{max width = 1.5\textheight}
\begin{tabular}{lccccccccccccccccccccc}
\hline \noalign{\smallskip} & \multicolumn{3}{c}{\uline{\hfill Female \hfill}} & \multicolumn{3}{c}{\uline{\hfill Youth \hfill}} & \multicolumn{3}{c}{\uline{\hfill Wealth \hfill}} & \multicolumn{3}{c}{\uline{\hfill Muslim minority \hfill}} & \multicolumn{3}{c}{\uline{\hfill Any ethnic minority \hfill}} & \multicolumn{3}{c}{\uline{\hfill Prior peace education \hfill}} & \multicolumn{3}{c}{\uline{\hfill Current member of a peace group \hfill}}\\
 & Treatment & Interaction & Sum & Treatment & Interaction & Sum & Treatment & Interaction & Sum & Treatment & Interaction & Sum & Treatment & Interaction & Sum & Treatment & Interaction & Sum & Treatment & Interaction & Sum\\
\noalign{\smallskip}\hline \noalign{\smallskip}Security rights index, aggregated & -0.093 & 0.015 & -0.078 & -0.087 & 0.003 & -0.084 & -0.088 & 0.067 & -0.020 & -0.091 & 0.050 & -0.041 & -0.090 & 0.035 & -0.054 & -0.119 & 0.081 & -0.038 & -0.213 & 0.469 & 0.256\\
 & \begin{footnotesize}[0.051]*\end{footnotesize} & \begin{footnotesize}[0.064]\end{footnotesize} & \begin{footnotesize}[0.047]*\end{footnotesize} & \begin{footnotesize}[0.045]*\end{footnotesize} & \begin{footnotesize}[0.066]\end{footnotesize} & \begin{footnotesize}[0.054]\end{footnotesize} & \begin{footnotesize}[0.037]**\end{footnotesize} & \begin{footnotesize}[0.060]\end{footnotesize} & \begin{footnotesize}[0.075]\end{footnotesize} & \begin{footnotesize}[0.040]**\end{footnotesize} & \begin{footnotesize}[0.098]\end{footnotesize} & \begin{footnotesize}[0.089]\end{footnotesize} & \begin{footnotesize}[0.040]**\end{footnotesize} & \begin{footnotesize}[0.099]\end{footnotesize} & \begin{footnotesize}[0.090]\end{footnotesize} & \begin{footnotesize}[0.090]\end{footnotesize} & \begin{footnotesize}[0.210]\end{footnotesize} & \begin{footnotesize}[0.133]\end{footnotesize} & \begin{footnotesize}[0.072]***\end{footnotesize} & \begin{footnotesize}[0.208]**\end{footnotesize} & \begin{footnotesize}[0.151]*\end{footnotesize}\\
\noalign{\smallskip}Improvement index, aggregated & -0.112 & 0.086 & -0.025 & -0.082 & 0.029 & -0.052 & -0.070 & 0.099 & 0.029 & -0.060 & -0.056 & -0.117 & -0.060 & -0.062 & -0.122 & -0.135 & 0.174 & 0.039 & -0.199 & 0.492 & 0.293\\
 & \begin{footnotesize}[0.055]**\end{footnotesize} & \begin{footnotesize}[0.063]\end{footnotesize} & \begin{footnotesize}[0.042]\end{footnotesize} & \begin{footnotesize}[0.052]\end{footnotesize} & \begin{footnotesize}[0.064]\end{footnotesize} & \begin{footnotesize}[0.047]\end{footnotesize} & \begin{footnotesize}[0.037]*\end{footnotesize} & \begin{footnotesize}[0.052]*\end{footnotesize} & \begin{footnotesize}[0.063]\end{footnotesize} & \begin{footnotesize}[0.039]\end{footnotesize} & \begin{footnotesize}[0.092]\end{footnotesize} & \begin{footnotesize}[0.091]\end{footnotesize} & \begin{footnotesize}[0.039]\end{footnotesize} & \begin{footnotesize}[0.092]\end{footnotesize} & \begin{footnotesize}[0.091]\end{footnotesize} & \begin{footnotesize}[0.075]*\end{footnotesize} & \begin{footnotesize}[0.164]\end{footnotesize} & \begin{footnotesize}[0.105]\end{footnotesize} & \begin{footnotesize}[0.066]***\end{footnotesize} & \begin{footnotesize}[0.221]**\end{footnotesize} & \begin{footnotesize}[0.171]*\end{footnotesize}\\
\noalign{\smallskip}Index of Fallow Land, farm & 0.001 & -0.012 & -0.011 & -0.053 & 0.096 & 0.043 & -0.007 & 0.079 & 0.071 & -0.012 & 0.060 & 0.048 & -0.011 & 0.053 & 0.042 & -0.018 & 0.035 & 0.017 & -0.088 & 0.309 & 0.221\\
 & \begin{footnotesize}[0.053]\end{footnotesize} & \begin{footnotesize}[0.068]\end{footnotesize} & \begin{footnotesize}[0.057]\end{footnotesize} & \begin{footnotesize}[0.056]\end{footnotesize} & \begin{footnotesize}[0.066]\end{footnotesize} & \begin{footnotesize}[0.053]\end{footnotesize} & \begin{footnotesize}[0.043]\end{footnotesize} & \begin{footnotesize}[0.063]\end{footnotesize} & \begin{footnotesize}[0.075]\end{footnotesize} & \begin{footnotesize}[0.046]\end{footnotesize} & \begin{footnotesize}[0.105]\end{footnotesize} & \begin{footnotesize}[0.097]\end{footnotesize} & \begin{footnotesize}[0.046]\end{footnotesize} & \begin{footnotesize}[0.104]\end{footnotesize} & \begin{footnotesize}[0.096]\end{footnotesize} & \begin{footnotesize}[0.103]\end{footnotesize} & \begin{footnotesize}[0.218]\end{footnotesize} & \begin{footnotesize}[0.131]\end{footnotesize} & \begin{footnotesize}[0.083]\end{footnotesize} & \begin{footnotesize}[0.229]\end{footnotesize} & \begin{footnotesize}[0.163]\end{footnotesize}\\
\noalign{\smallskip}Size of farm & 5.609 & -3.734 & 1.875 & 5.063 & -2.654 & 2.409 & 3.614 & 2.553 & 6.167 & 3.498 & 2.138 & 5.635 & 3.520 & 1.878 & 5.398 & 3.399 & 0.888 & 4.287 & 5.468 & -7.376 & -1.908\\
 & \begin{footnotesize}[2.225]**\end{footnotesize} & \begin{footnotesize}[2.875]\end{footnotesize} & \begin{footnotesize}[1.837]\end{footnotesize} & \begin{footnotesize}[2.173]**\end{footnotesize} & \begin{footnotesize}[2.761]\end{footnotesize} & \begin{footnotesize}[1.809]\end{footnotesize} & \begin{footnotesize}[1.415]**\end{footnotesize} & \begin{footnotesize}[2.791]\end{footnotesize} & \begin{footnotesize}[3.350]*\end{footnotesize} & \begin{footnotesize}[1.523]**\end{footnotesize} & \begin{footnotesize}[4.357]\end{footnotesize} & \begin{footnotesize}[4.147]\end{footnotesize} & \begin{footnotesize}[1.527]**\end{footnotesize} & \begin{footnotesize}[4.385]\end{footnotesize} & \begin{footnotesize}[4.149]\end{footnotesize} & \begin{footnotesize}[3.044]\end{footnotesize} & \begin{footnotesize}[7.497]\end{footnotesize} & \begin{footnotesize}[5.011]\end{footnotesize} & \begin{footnotesize}[2.895]*\end{footnotesize} & \begin{footnotesize}[10.319]\end{footnotesize} & \begin{footnotesize}[7.964]\end{footnotesize}\\
\noalign{\smallskip}\hline\end{tabular}

\end{adjustbox}
\end{center}
\end{table}


\begin{table}[H]
\caption{Summary statistics of key outcomes by demographics and peace background}
\label{summary_stats_demo}
\begin{center}
\begin{adjustbox}{max width = 1.5\textheight}
\begin{tabular}{lcccccccccccc}
\hline \noalign{\smallskip} & \multicolumn{2}{c}{\uline{\hfill All Endline 2 residents}} & \multicolumn{2}{c}{\uline{\hfill Gender \hfill}} & \multicolumn{2}{c}{\uline{\hfill Age \hfill}} & \multicolumn{2}{c}{\uline{\hfill Muslim minority \hfill}} & \multicolumn{2}{c}{\uline{\hfill Any ethnic minority \hfill}} & \multicolumn{2}{c}{\uline{\hfill Any town Prior peace education \hfill}}\\
 & Mean & SD & Men & Women & above 40 & 20-40 & No & Yes & No & Yes & No & Yes\\
\noalign{\smallskip}\hline \noalign{\smallskip}\textbf{Pct. of Endline 2 residents} &  &  & 0.48 & 0.52 & 0.49 & 0.51 & 0.87 & 0.13 & 0.87 & 0.13 & 0.03 & 0.97\\
\textbf{Panel A: Land dispute outomes for all residents} &  &  &  &  &  &  &  &  &  &  &  & \\
Any serious land dispute & 0.09 & 0.28 & 0.10 & 0.08 & 0.08 & 0.10 & 0.09 & 0.08 & 0.09 & 0.08 & 0.12 & 0.09\\
Any unresolved land dispute & 0.02 & 0.15 & 0.03 & 0.02 & 0.02 & 0.03 & 0.02 & 0.02 & 0.02 & 0.02 & 0.04 & 0.02\\
Any threats, property damage, or violence & 0.04 & 0.18 & 0.04 & 0.03 & 0.03 & 0.04 & 0.04 & 0.02 & 0.04 & 0.03 & 0.02 & 0.04\\
\quad Property damage or violence in land dispute & 0.02 & 0.13 & 0.02 & 0.01 & 0.01 & 0.02 & 0.02 & 0.02 & 0.02 & 0.02 & 0.01 & 0.02\\
\tab Threats & 0.03 & 0.17 & 0.03 & 0.03 & 0.03 & 0.03 & 0.03 & 0.02 & 0.03 & 0.02 & 0.02 & 0.03\\
\tab Property damage & 0.01 & 0.09 & 0.01 & 0.01 & 0.01 & 0.01 & 0.01 & 0.01 & 0.01 & 0.01 & 0.00 & 0.01\\
\tab Violence & 0.02 & 0.13 & 0.02 & 0.01 & 0.01 & 0.02 & 0.02 & 0.02 & 0.02 & 0.02 & 0.01 & 0.02\\
\textbf{Panel B: Conditional on a land dispute} &  &  &  &  &  &  &  &  &  &  &  & \\
Length of maximum land conflict & 14.50 & 28.07 & 14.95 & 13.95 & 16.20 & 13.11 & 14.30 & 16.02 & 14.35 & 15.63 & 14.49 & 14.50\\
Resolved land dispute & 0.66 & 0.47 & 0.65 & 0.67 & 0.69 & 0.64 & 0.66 & 0.70 & 0.66 & 0.68 & 0.67 & 0.66\\
Any threats, property damage, or violence \phantom{} & 0.40 & 0.49 & 0.41 & 0.40 & 0.36 & 0.44 & 0.42 & 0.30 & 0.41 & 0.32 & 0.20 & 0.41\\
\quad Property damage or violence \tab & 0.21 & 0.41 & 0.23 & 0.18 & 0.16 & 0.24 & 0.21 & 0.20 & 0.21 & 0.22 & 0.07 & 0.21\\
\tab Threats \phantom{} & 0.35 & 0.48 & 0.35 & 0.36 & 0.33 & 0.37 & 0.36 & 0.25 & 0.37 & 0.24 & 0.13 & 0.36\\
\tab Property damage \phantom{} & 0.09 & 0.29 & 0.10 & 0.08 & 0.07 & 0.11 & 0.10 & 0.07 & 0.09 & 0.10 & 0.00 & 0.10\\
\tab Violence \phantom{} & 0.18 & 0.39 & 0.20 & 0.16 & 0.14 & 0.22 & 0.18 & 0.20 & 0.18 & 0.20 & 0.07 & 0.19\\
\tab Witchcraft \phantom{} & 0.07 & 0.26 & 0.08 & 0.06 & 0.03 & 0.10 & 0.08 & 0.03 & 0.08 & 0.02 & 0.00 & 0.07\\
\textbf{Panel C: General dispute outcomes for all residents} &  &  &  &  &  &  &  &  &  &  &  & \\
Any serious dispute & 0.31 & 0.46 & 0.31 & 0.30 & 0.25 & 0.36 & 0.32 & 0.24 & 0.32 & 0.24 & 0.35 & 0.30\\
Any unresolved dispute & 0.06 & 0.25 & 0.07 & 0.06 & 0.06 & 0.07 & 0.06 & 0.06 & 0.06 & 0.06 & 0.12 & 0.06\\
\phantom{} Any threats, property damage, or violence & 0.09 & 0.29 & 0.09 & 0.10 & 0.07 & 0.11 & 0.10 & 0.05 & 0.10 & 0.05 & 0.09 & 0.09\\
\textbf{Panel D: Conditional on a dispute} &  &  &  &  &  &  &  &  &  &  &  & \\
Resolved dispute & 0.76 & 0.43 & 0.78 & 0.74 & 0.74 & 0.77 & 0.76 & 0.76 & 0.76 & 0.75 & 0.75 & 0.76\\
\quad Resolved via informal mechanism & 0.41 & 0.49 & 0.40 & 0.41 & 0.39 & 0.41 & 0.41 & 0.37 & 0.41 & 0.37 & 0.41 & 0.40\\
\phantom{} Any threats, property damage, or violence \phantom{} & 0.31 & 0.46 & 0.29 & 0.33 & 0.30 & 0.31 & 0.32 & 0.20 & 0.32 & 0.21 & 0.25 & 0.31\\
\quad Property damage or violence \phantom{} & 0.18 & 0.39 & 0.16 & 0.20 & 0.16 & 0.20 & 0.19 & 0.11 & 0.19 & 0.12 & 0.11 & 0.19\\
\quad Threats \phantom{} & 0.25 & 0.44 & 0.24 & 0.27 & 0.25 & 0.25 & 0.26 & 0.17 & 0.26 & 0.17 & 0.23 & 0.25\\
\quad Property damage \phantom{} & 0.04 & 0.20 & 0.05 & 0.03 & 0.04 & 0.04 & 0.04 & 0.03 & 0.04 & 0.04 & 0.02 & 0.04\\
\quad Violence \phantom{} & 0.17 & 0.38 & 0.15 & 0.19 & 0.15 & 0.19 & 0.18 & 0.11 & 0.18 & 0.11 & 0.11 & 0.17\\
\textbf{Panel E: Security and investment} &  &  &  &  &  &  &  &  &  &  &  & \\
Security rights index for house and farm & -0.00 & 1.00 & 0.12 & -0.11 & 0.05 & -0.05 & 0.04 & -0.25 & 0.04 & -0.26 & -0.32 & 0.01\\
Improvement index for house and farm & 0.00 & 1.00 & 0.16 & -0.15 & -0.02 & 0.02 & 0.01 & -0.05 & 0.01 & -0.05 & -0.08 & 0.00\\
Index of Fallow Land for farm & -0.00 & 1.00 & 0.09 & -0.09 & 0.04 & -0.04 & -0.01 & 0.10 & -0.02 & 0.10 & 0.09 & -0.00\\
Size of farm & 38.43 & 56.85 & 43.99 & 32.92 & 43.42 & 33.41 & 39.03 & 34.09 & 39.04 & 34.06 & 36.00 & 38.51\\
\noalign{\smallskip}\hline\end{tabular}

\end{adjustbox}
\end{center}
\end{table}
\end{landscape}

\begin{table}[H]
\caption{Questions on Attitudes, Norms and Skills}
\begin{center}
\begin{adjustbox}{max width = 1.5\textheight}
\end{adjustbox}
\end{center}
\end{table}

\section{Extra Tables}
\clearpage
\setcounter{table}{0}   
\renewcommand{\thetable}{E.\arabic{table}}

\subsection{All conflict and security variables}

\begin{table}[H]
\caption{Additional p-value corrections, general and land conflict combined}
\label{conflict_adj_p_year}
\begin{center}
\begin{adjustbox}{max width = \textwidth}
\begin{tabular}{lcccccccccccccc}
\hline \noalign{\smallskip} & \multicolumn{7}{c}{\uline{\hfill 1-year endline \hfill}} & \multicolumn{7}{c}{\uline{\hfill 3-year endline \hfill}}\\
 &  &  &  & ITT / &  &  &  &  &  &  & ITT / &  &  & \\
 & Control &  &  & control & Est. & WY Adj. & Holms Adj & Control &  &  & control & Est. & WY Adj. & Holms Adj\\
Dependent Variable & mean & N & ITT & mean (\%) & p-val & p-val & p-val & mean & N & ITT & mean (\%) & p-val & p-val & p-val\\
 & (1) & (2) & (3) & (4) & (5) & (6) & (7) & (8) & (9) & (10) & (11) & (12) & (13) & (14)\\
\noalign{\smallskip}\hline \noalign{\smallskip}\textbf{Panel A: Land dispute outomes for all residents} &  &  &  &  &  &  &  &  &  &  &  &  &  & \\
 &  &  &  &  &  &  &  &  &  &  &  &  &  & \\
Any serious land dispute & 0.221 & 5,435 & 0.003 & 1.2 & 0.867 & 0.864\textsuperscript{a} & 0.868 & 0.087 & 4,011 & 0.008 & 8.8 & 0.473 & 0.984\textsuperscript{b} & 0.986\\
 &  &  & [0.016] &  &  &  &  &  &  & [0.011] &  &  &  & \\
Any unresolved land dispute & 0.070 & 5,435 & -0.020 & -28.0 & 0.015 & 0.148\textsuperscript{a} & 0.111 & 0.024 & 4,011 & 0.002 & 6.4 & 0.744 & 0.984\textsuperscript{b} & 0.986\\
 &  &  & [0.008]** &  &  &  &  &  &  & [0.005] &  &  &  & \\
Any threats, property damage, or violence & 0.122 & 5,435 & -0.010 & -8.1 & 0.397 & 0.823\textsuperscript{a} & 0.868 & 0.041 & 4,011 & -0.012 & -29.3 & 0.039 & 0.450\textsuperscript{b} & 0.331\\
 &  &  & [0.012] &  &  &  &  &  &  & [0.006]** &  &  &  & \\
\quad Property damage or violence in land dispute & 0.091 & 5,435 & -0.012 & -13.2 & 0.183 &  &  & 0.021 & 4,011 & -0.007 & -31.2 & 0.117 &  & \\
 &  &  & [0.009] &  &  &  &  &  &  & [0.004] &  &  &  & \\
\tab Threats & 0.114 & 5,435 & -0.006 & -5.3 & 0.596 &  &  & 0.035 & 4,011 & -0.010 & -28.9 & 0.069 &  & \\
 &  &  & [0.011] &  &  &  &  &  &  & [0.006]* &  &  &  & \\
\tab Property damage & 0.041 & 5,435 & -0.013 & -32.4 & 0.029 &  &  & 0.010 & 4,011 & -0.005 & -52.3 & 0.072 &  & \\
 &  &  & [0.006]** &  &  &  &  &  &  & [0.003]* &  &  &  & \\
\tab Violence & 0.077 & 5,435 & -0.007 & -8.7 & 0.416 &  &  & 0.017 & 4,011 & -0.004 & -21.4 & 0.361 &  & \\
 &  &  & [0.008] &  &  &  &  &  &  & [0.004] &  &  &  & \\
\textbf{Panel B: Conditional on a land dispute} &  &  &  &  &  &  &  &  &  &  &  &  &  & \\
 &  &  &  &  &  &  &  &  &  &  &  &  &  & \\
Length of maximum land conflict &  &  &  &  &  &  &  & 13.247 & 353 & 3.642 & 27.5 & 0.209 & 0.871\textsuperscript{b} & 0.847\\
 &  &  &  &  &  &  &  &  &  & [2.889] &  &  &  & \\
Resolved land dispute & 0.684 & 1,212 & 0.072 & 10.5 & 0.009 & 0.109\textsuperscript{a} & 0.080 & 0.668 & 353 & -0.024 & -3.6 & 0.604 & 0.984\textsuperscript{b} & 0.986\\
 &  &  & [0.027]*** &  &  &  &  &  &  & [0.046] &  &  &  & \\
Any threats, property damage, or violence \phantom{} & 0.554 & 1,212 & -0.024 & -4.3 & 0.496 & 0.856\textsuperscript{a} & 0.868 & 0.476 & 353 & -0.192 & -40.4 & 0.000 & 0.006\textsuperscript{b} & 0.001\\
 &  &  & [0.035] &  &  &  &  &  &  & [0.047]*** &  &  &  & \\
\quad Property damage or violence \tab & 0.411 & 1,212 & -0.037 & -9.1 & 0.213 &  &  & 0.243 & 353 & -0.090 & -37.2 & 0.035 &  & \\
 &  &  & [0.030] &  &  &  &  &  &  & [0.042]** &  &  &  & \\
\tab Threats \phantom{} & 0.515 & 1,212 & -0.013 & -2.5 & 0.713 &  &  & 0.408 & 353 & -0.157 & -38.6 & 0.001 &  & \\
 &  &  & [0.035] &  &  &  &  &  &  & [0.048]*** &  &  &  & \\
\tab Property damage \phantom{} & 0.186 & 1,212 & -0.051 & -27.4 & 0.041 &  &  & 0.114 & 353 & -0.067 & -58.5 & 0.016 &  & \\
 &  &  & [0.025]** &  &  &  &  &  &  & [0.027]** &  &  &  & \\
\tab Violence \phantom{} & 0.349 & 1,212 & -0.022 & -6.3 & 0.442 &  &  & 0.202 & 353 & -0.056 & -28.0 & 0.186 &  & \\
 &  &  & [0.028] &  &  &  &  &  &  & [0.042] &  &  &  & \\
\tab Witchcraft \phantom{} &  &  &  &  &  &  &  & 0.065 & 353 & 0.035 & 54.3 & 0.182 &  & \\
 &  &  &  &  &  &  &  &  &  & [0.026] &  &  &  & \\
\textbf{Panel C: General dispute outcomes for all residents} &  &  &  &  &  &  &  &  &  &  &  &  &  & \\
 &  &  &  &  &  &  &  &  &  &  &  &  &  & \\
Any serious dispute & 0.299 & 5,435 & 0.022 & 7.2 & 0.220 & 0.702\textsuperscript{a} & 0.775 & 0.306 & 4,011 & 0.012 & 4.1 & 0.456 & 0.984\textsuperscript{b} & 0.986\\
 &  &  & [0.018] &  &  &  &  &  &  & [0.017] &  &  &  & \\
Any unresolved dispute & 0.118 & 5,435 & -0.013 & -11.0 & 0.250 & 0.720\textsuperscript{a} & 0.775 & 0.064 & 4,011 & -0.004 & -6.6 & 0.623 & 0.984\textsuperscript{b} & 0.986\\
 &  &  & [0.011] &  &  &  &  &  &  & [0.009] &  &  &  & \\
\phantom{} Any threats, property damage, or violence &  &  &  &  &  &  &  & 0.101 & 4,011 & -0.015 & -15.2 & 0.111 & 0.718\textsuperscript{b} & 0.652\\
 &  &  &  &  &  &  &  &  &  & [0.010] &  &  &  & \\
\textbf{Panel D: Conditional on a dispute} &  &  &  &  &  &  &  &  &  &  &  &  &  & \\
 &  &  &  &  &  &  &  &  &  &  &  &  &  & \\
Resolved dispute & 0.676 & 1,670 & 0.051 & 7.5 & 0.033 & 0.235\textsuperscript{a} & 0.208 & 0.767 & 1,227 & -0.019 & -2.5 & 0.467 & 0.984\textsuperscript{b} & 0.986\\
 &  &  & [0.024]** &  &  &  &  &  &  & [0.026] &  &  &  & \\
\quad Resolved via informal mechanism & 0.247 & 1,671 & 0.018 & 7.4 & 0.470 & 0.856\textsuperscript{a} & 0.868 & 0.409 & 1,227 & -0.019 & -4.6 & 0.487 & 0.984\textsuperscript{b} & 0.986\\
 &  &  & [0.025] &  &  &  &  &  &  & [0.027] &  &  &  & \\
\phantom{} Any threats, property damage, or violence \phantom{} &  &  &  &  &  &  &  & 0.331 & 1,227 & -0.069 & -20.7 & 0.009 & 0.194\textsuperscript{b} & 0.095\\
 &  &  &  &  &  &  &  &  &  & [0.026]*** &  &  &  & \\
\quad Property damage or violence \phantom{} &  &  &  &  &  &  &  & 0.190 & 1,227 & -0.027 & -14.1 & 0.222 &  & \\
 &  &  &  &  &  &  &  &  &  & [0.022] &  &  &  & \\
\quad Threats \phantom{} &  &  &  &  &  &  &  & 0.274 & 1,227 & -0.072 & -26.2 & 0.003 &  & \\
 &  &  &  &  &  &  &  &  &  & [0.024]*** &  &  &  & \\
\quad Property damage \phantom{} &  &  &  &  &  &  &  & 0.048 & 1,227 & -0.019 & -40.0 & 0.085 &  & \\
 &  &  &  &  &  &  &  &  &  & [0.011]* &  &  &  & \\
\quad Violence \phantom{} &  &  &  &  &  &  &  & 0.173 & 1,227 & -0.018 & -10.2 & 0.411 &  & \\
 &  &  &  &  &  &  &  &  &  & [0.021] &  &  &  & \\
\noalign{\smallskip}\hline\end{tabular}

\end{adjustbox}
\end{center}
\end{table}


\begin{table}[H]
\caption{All land conflict results, endlines 1 and 2}
\begin{center}
\begin{adjustbox}{max width = \textwidth}
\begin{tabular}{lcccccccc}

\end{adjustbox}
\end{center}
\end{table}

\begin{table}[H]
\caption{All general conflict results, endlines 1 and 2}
\begin{center}
\begin{adjustbox}{max width = \textwidth}
\begin{tabular}{lcccccccc}
\hline \noalign{\smallskip} & \multicolumn{4}{c}{\uline{\hfill 1-year endline \hfill}} & \multicolumn{4}{c}{\uline{\hfill 3-year endline \hfill}}\\
 &  &  &  & ITT /  &  &  &  & ITT /\\
 & Control &  &  & control & Control &  &  & control\\
Dependent Variable & mean & N & ITT & mean (\%) & mean & N & ITT & mean (\%)\\
 & (1) & (2) & (3) & (4) & (5) & (6) & (7) & (8)\\
\noalign{\smallskip}\hline \noalign{\smallskip}\textbf{Panel A: Ouctomes for all residents} &  &  &  &  &  &  &  & \\
 &  &  &  &  &  &  &  & \\
Any serious dispute & 0.299 & 5,435 & 0.022 & 7.2 & 0.306 & 4,011 & 0.012 & 4.1\\
 &  &  & [0.018] &  &  &  & [0.017] & \\
Any unresolved dispute & 0.118 & 5,435 & -0.013 & -11.0 & 0.064 & 4,011 & -0.004 & -6.6\\
 &  &  & [0.011] &  &  &  & [0.009] & \\
\phantom{} Any threats, property damage, or violence &  &  &  &  & 0.101 & 4,011 & -0.015 & -15.2\\
 &  &  &  &  &  &  & [0.010] & \\
\quad Property damage or violence in a land dispute &  &  &  &  & 0.058 & 4,011 & -0.006 & -10.8\\
 &  &  &  &  &  &  & [0.008] & \\
\quad Threats &  &  &  &  & 0.084 & 4,011 & -0.018 & -20.9\\
 &  &  &  &  &  &  & [0.009]** & \\
\quad Property damage &  &  &  &  & 0.015 & 4,011 & -0.006 & -37.2\\
 &  &  &  &  &  &  & [0.004] & \\
\quad Violence &  &  &  &  & 0.053 & 4,011 & -0.003 & -4.8\\
 &  &  &  &  &  &  & [0.007] & \\
Length of maximum conflict &  &  &  &  & 1.852 & 4,011 & 0.447 & 24.1\\
 &  &  &  &  &  &  & [0.289] & \\
\quad Length of mean conflict &  &  &  &  & 0.400 & 4,011 & 0.081 & 20.1\\
 &  &  &  &  &  &  & [0.060] & \\
\quad Length of mean conflict \phantom{} &  &  &  &  & 1.308 & 1,227 & 0.201 & 15.4\\
 &  &  &  &  &  &  & [0.161] & \\
Length of maximum conflict \phantom{} &  &  &  &  & 6.052 & 1,227 & 1.164 & 19.2\\
 &  &  &  &  &  &  & [0.788] & \\
\textbf{Panel B: Conditional on dispute} &  &  &  &  &  &  &  & \\
 &  &  &  &  &  &  &  & \\
Resolved dispute & 0.676 & 1,670 & 0.051 & 7.5 & 0.767 & 1,227 & -0.019 & -2.5\\
 &  &  & [0.024]** &  &  &  & [0.026] & \\
\quad Resolved via informal mechanism & 0.247 & 1,671 & 0.018 & 7.4 & 0.409 & 1,227 & -0.019 & -4.6\\
 &  &  & [0.025] &  &  &  & [0.027] & \\
\phantom{} Any threats, property damage, or violence \phantom{} &  &  &  &  & 0.331 & 1,227 & -0.069 & -20.7\\
 &  &  &  &  &  &  & [0.026]*** & \\
\quad Property damage or violence \phantom{} &  &  &  &  & 0.190 & 1,227 & -0.027 & -14.1\\
 &  &  &  &  &  &  & [0.022] & \\
\quad Threats \phantom{} &  &  &  &  & 0.274 & 1,227 & -0.072 & -26.2\\
 &  &  &  &  &  &  & [0.024]*** & \\
\quad Property damage \phantom{} &  &  &  &  & 0.048 & 1,227 & -0.019 & -40.0\\
 &  &  &  &  &  &  & [0.011]* & \\
\quad Violence \phantom{} &  &  &  &  & 0.173 & 1,227 & -0.018 & -10.2\\
 &  &  &  &  &  &  & [0.021] & \\
\quad Witchcraft \phantom{} &  &  &  &  & 0.029 & 1,227 & 0.006 & 19.7\\
 &  &  &  &  &  &  & [0.010] & \\
\noalign{\smallskip}\hline\end{tabular}

\end{adjustbox}
\end{center}
\end{table}

\begin{table}[H]
\caption{All land and security results}
\begin{center}
\begin{adjustbox}{max width = \textwidth}
\begin{tabular}{lcccc}
\hline \noalign{\smallskip} &  &  &  & ITT /\\
 &  & Control &  & control\\
Dependent Variable & N & mean & ITT & mean (\%)\\
 & (1) & (2) & (3) & (4)\\
\noalign{\smallskip}\hline \noalign{\smallskip}Security rights index, z-score & 4,011 & 0.045 & -0.085 & \\
 &  &  & [0.037]** & \\
\quad Security rights index for farm & 3,666 & 0.048 & -0.090 & \\
 &  &  & [0.037]** & \\
\quad Security rights index for house & 3,850 & 0.034 & -0.068 & \\
 &  &  & [0.038]* & \\
Improvement index, z-score & 4,011 & 0.023 & -0.066 & \\
 &  &  & [0.037]* & \\
\quad Improvement index for farm & 4,011 & 0.015 & -0.061 & \\
 &  &  & [0.034]* & \\
\quad Improvement index for house & 3,850 & 0.022 & -0.042 & \\
 &  &  & [0.034] & \\
Index of Fallow Land for farm & 3,666 & 0.003 & -0.004 & \\
 &  &  & [0.044] & \\
\quad Has ever let land rest & 3,666 & 0.682 & 0.018 & 2.6\\
 &  &  & [0.020] & \\
\quad Will let land rest in future & 3,666 & 0.730 & 0.014 & 1.9\\
 &  &  & [0.018] & \\
\quad Num. of seasons to rest & 3,666 & 3.347 & -0.063 & -1.9\\
 &  &  & [0.145] & \\
\quad Level of security if land resting & 3,666 & 2.322 & -0.052 & -2.2\\
 &  &  & [0.036] & \\
Size of farm & 3,598 & 37.481 & 2.773 & 7.4\\
 &  &  & [1.051]*** & \\
\noalign{\smallskip}\hline\end{tabular}

\end{adjustbox}
\end{center}
\end{table}



\subsection{Subcomponents of Norms and Skills}

\begin{table}[H]
\caption{Resident-Level Effect of Treatment Assignment Bias}
\begin{center}
\begin{adjustbox}{max width = \textwidth}
\begin{tabular}{lcccccc}
\hline \noalign{\smallskip} &  &  &  &  &  & \\
 &  & Control &  & Est. & Wy adj. & Holms adj.\\
Dependent Variable & N & mean & ITT & p-val & p-val & p-val\\
 & (1) & (2) & (3) & (4) & (5) & (6)\\
\noalign{\smallskip}\hline \noalign{\smallskip}\quad Bias index & 4,011 & -0.009 & -0.002 & 0.962 & 1.000 & 0.999\\
 &  &  & [0.046] &  &  & \\
\quad Thinks bad to gossip (z-score) & 4,011 & 0.039 & -0.072 & 0.068 & 0.000 & 0.390\\
 &  &  & [0.039]* &  &  & \\
\quad Comm. thinks bad to gossip (z-score) & 3,721 & -0.028 & 0.026 & 0.555 & 1.000 & 0.961\\
 &  &  & [0.043] &  &  & \\
\quad Thinks bad to tell lies (z-score) & 4,011 & 0.003 & -0.001 & 0.984 & 1.000 & 0.999\\
 &  &  & [0.038] &  &  & \\
\quad Comm. thinks bad to tell lies (z-score) & 3,717 & -0.046 & 0.057 & 0.126 & 0.500 & 0.554\\
 &  &  & [0.037] &  &  & \\
\quad Thinks bad to take sides (z-score) & 4,011 & 0.026 & -0.048 & 0.219 & 0.500 & 0.710\\
 &  &  & [0.039] &  &  & \\
\quad Comm thinks bad to take sides (z-score) & 3,833 & -0.034 & 0.023 & 0.595 & 1.000 & 0.961\\
 &  &  & [0.043] &  &  & \\
\noalign{\smallskip}\hline\end{tabular}

\end{adjustbox}
\end{center}
\end{table}

\begin{table}[H]
\caption{Resident-Level Effect of Treatment Assignment on Defection}
\begin{center}
\begin{adjustbox}{max width = \textwidth}
\begin{tabular}{lccc}
\hline \noalign{\smallskip} &  &  & \\
 &  & Control & \\
Dependent Variable & N & mean & ITT\\
 & (1) & (2) & (3)\\
\noalign{\smallskip}\hline \noalign{\smallskip}\quad Defection index & 4,011 & -0.043 & 0.045\\
 &  &  & [0.041]\\
\quad Thinks bad to renege (z-score) & 4,011 & 0.014 & -0.061\\
 &  &  & [0.045]\\
\quad Comm. thinks bad to renege (z-score) & 3,717 & 0.000 & -0.066\\
 &  &  & [0.042]\\
\quad Thinks good to take palava to elders (z-score) & 4,011 & -0.044 & 0.067\\
 &  &  & [0.036]*\\
\quad Comm. thinks good to take palava to elders (z-score) & 3,706 & -0.027 & 0.016\\
 &  &  & [0.038]\\
\quad Doesn't think okay to take palava to police (z-score) & 4,011 & 0.005 & 0.009\\
 &  &  & [0.037]\\
\quad Comm. doesn't think okay to take palava to police (z-score) & 3,846 & -0.043 & 0.066\\
 &  &  & [0.046]\\
\quad Thinks bad to take palava to biased forum (z-score) & 4,011 & -0.021 & 0.038\\
 &  &  & [0.038]\\
\quad Comm thinks bad to take palava to biased forum (z-score) & 3,693 & -0.058 & 0.088\\
 &  &  & [0.039]**\\
\noalign{\smallskip}\hline\end{tabular}

\end{adjustbox}
\end{center}
\end{table}

\begin{table}[H]
\caption{Resident-Level Effect of Treatment Assignment on Forum Choice}
\begin{center}
\begin{tabular}{lcccccc}
\hline \noalign{\smallskip} &  &  &  &  &  & \\
 &  & Control &  & Est. & Wy adj. & Holms adj.\\
Dependent Variable & N & mean & ITT & p-val & p-val & p-val\\
 & (1) & (2) & (3) & (4) & (5) & (6)\\
\noalign{\smallskip}\hline \noalign{\smallskip}\quad Forum choice index & 4,011 & -0.028 & 0.031 & 0.400 & 0.903 & 0.908\\
 &  &  & [0.037] &  &  & \\
\quad Thinks good to go to good talkers first (z-score) & 4,011 & -0.004 & -0.005 & 0.891 & 0.990 & 0.988\\
 &  &  & [0.038] &  &  & \\
\quad Thinks good to talk to others first (z-score) & 3,850 & -0.013 & 0.046 & 0.222 & 0.819 & 0.828\\
 &  &  & [0.038] &  &  & \\
\quad Thinks good to talk (z-score) & 4,011 & 0.002 & -0.030 & 0.380 & 0.903 & 0.908\\
 &  &  & [0.034] &  &  & \\
\quad Comm. thinks good to talk (z-score) & 3,926 & -0.003 & 0.001 & 0.988 & 0.990 & 0.988\\
 &  &  & [0.035] &  &  & \\
\quad Doesn't think okay to go to police first (z-score) & 4,011 & -0.013 & 0.018 & 0.640 & 0.970 & 0.953\\
 &  &  & [0.038] &  &  & \\
\quad Comm doesn't think okay to go to police first (z-score) & 3,824 & -0.050 & 0.047 & 0.273 & 0.855 & 0.853\\
 &  &  & [0.043] &  &  & \\
\noalign{\smallskip}\hline\end{tabular}

\end{center}
\end{table}


\begin{table}[H]
\caption{Resident-Level Effect of Treatment Assignment on Managing Emotions}
\begin{center}
\begin{adjustbox}{max width = \textwidth}
\begin{tabular}{lcccccc}
\hline \noalign{\smallskip} &  &  &  &  &  & \\
 &  & Control &  & Est. & Wy adj. & Holms adj.\\
Dependent Variable & N & mean & ITT & p-val & p-val & p-val\\
 & (1) & (2) & (3) & (4) & (5) & (6)\\
\noalign{\smallskip}\hline \noalign{\smallskip}\quad Managing emotions index & 4,011 & -0.031 & 0.067 & 0.032 & 0.215 & 0.150\\
 &  &  & [0.031]** &  &  & \\
\quad Thinks shouldn't spoil property (z-score) & 4,011 & 0.035 & -0.077 & 0.040 & 0.226 & 0.150\\
 &  &  & [0.037]** &  &  & \\
\quad Comm thinks shouldn't spoil property (z-score) & 3,760 & -0.043 & 0.076 & 0.067 & 0.260 & 0.187\\
 &  &  & [0.041]* &  &  & \\
\quad Stays calm (z-score) & 4,010 & -0.022 & 0.043 & 0.184 & 0.361 & 0.335\\
 &  &  & [0.032] &  &  & \\
\quad Cools temper (z-score) & 4,010 & -0.028 & 0.039 & 0.221 & 0.361 & 0.335\\
 &  &  & [0.032] &  &  & \\
\quad Don't talk bad (z-score) & 4,010 & -0.034 & 0.089 & 0.017 & 0.153 & 0.097\\
 &  &  & [0.037]** &  &  & \\
\noalign{\smallskip}\hline\end{tabular}

\end{adjustbox}
\end{center}
\end{table}

\begin{table}[H]
\caption{Resident-Level Effect of Treatment Assignment on Mediation}
\begin{center}
\begin{adjustbox}{max width = \textwidth}
\begin{tabular}{lccc}
\hline \noalign{\smallskip} &  &  & \\
 &  & Control & \\
Dependent Variable & N & mean & ITT\\
 & (1) & (2) & (3)\\
\noalign{\smallskip}\hline \noalign{\smallskip}mediation_index_ec2 & 4,011 & 0.003 & -0.062\\
 &  &  & [0.037]\\
What do you think? When you have palava with someone, it's okay to ask your frie & 4,011 & 0.007 & -0.037\\
 &  &  & [0.034]\\
In this community, how many people ask their friends or neighbors to help them a & 3,927 & -0.008 & -0.005\\
 &  &  & [0.035]\\
What do you think? When other people have palava, it's good to bring them togeth & 4,011 & 0.035 & -0.107\\
 &  &  & [0.040]\\
In this community, how many people will bring persons who have palava together t & 3,949 & -0.017 & -0.013\\
 &  &  & [0.033]\\
What do you think? It's okay to help two neighbors in palava find a solution, ev & 4,011 & 0.011 & -0.061\\
 &  &  & [0.041]\\
In this community, how many people can help two neighbors in palava find a solut & 3,953 & -0.021 & -0.001\\
 &  &  & [0.036]\\
When people make palava, some people like to get involved and others like to sta & 4,010 & -0.009 & -0.010\\
 &  &  & [0.035]\\
Since 2012, when some people have palava, how often do you help them reach agree & 4,010 & 0.009 & -0.050\\
 &  &  & [0.033]\\
Since 2012, how many times do you help other people understand each other?, z-sc & 4,010 & 0.002 & -0.018\\
 &  &  & [0.033]\\
Since 2012, when some other people make palava, how many times do you listen to  & 4,010 & -0.005 & -0.021\\
 &  &  & [0.035]\\
\noalign{\smallskip}\hline\end{tabular}

\end{adjustbox}
\end{center}
\end{table}

\begin{table}[H]
\caption{Resident-Level Effect of Treatment Assignment on Negotiation}
\begin{center}
\begin{adjustbox}{max width = \textwidth}
\begin{tabular}{lcccccc}
\hline \noalign{\smallskip} &  &  &  &  &  & \\
 &  & Control &  & Est. & Wy adj. & Holms adj.\\
Dependent Variable & N & mean & ITT & p-val & p-val & p-val\\
 & (1) & (2) & (3) & (4) & (5) & (6)\\
\noalign{\smallskip}\hline \noalign{\smallskip}\quad Negotiation index & 4,011 & 0.002 & 0.002 & 0.945 & 1.000 & 0.999\\
 &  &  & [0.027] &  &  & \\
\quad Comm. thinks its good to talk (z-score) & 4,011 & -0.004 & -0.011 & 0.758 & 1.000 & 0.999\\
 &  &  & [0.037] &  &  & \\
\quad Thinks its good to talk (z-score) & 3,919 & -0.002 & -0.032 & 0.310 & 1.000 & 0.965\\
 &  &  & [0.031] &  &  & \\
\quad Thinks gifts okay (z-score) & 4,011 & 0.011 & -0.006 & 0.865 & 1.000 & 0.999\\
 &  &  & [0.036] &  &  & \\
\quad Comm. thinks gifts okay (z-score) & 3,820 & 0.010 & -0.023 & 0.473 & 1.000 & 0.994\\
 &  &  & [0.032] &  &  & \\
\quad Talks with opposing side (z-score) & 4,010 & -0.008 & 0.009 & 0.787 & 1.000 & 0.999\\
 &  &  & [0.032] &  &  & \\
\quad Makes other side understand (z-score) & 4,010 & 0.005 & -0.003 & 0.923 & 1.000 & 0.999\\
 &  &  & [0.029] &  &  & \\
\quad Proposes solution (z-score) & 4,010 & -0.010 & 0.013 & 0.711 & 1.000 & 0.999\\
 &  &  & [0.035] &  &  & \\
\quad Compromizes with other (z-score) & 4,010 & -0.035 & 0.042 & 0.180 & 1.000 & 0.862\\
 &  &  & [0.031] &  &  & \\
\quad Forgives others (z-score) & 4,010 & -0.004 & 0.019 & 0.643 & 1.000 & 0.999\\
 &  &  & [0.040] &  &  & \\
\noalign{\smallskip}\hline\end{tabular}

\end{adjustbox}
\end{center}
\end{table}

\subsection{Using years-of-treatment as treatment variable}

\begin{landscape}
\begin{table}[H]
\caption{Resident-Level Effect of years since Treatment on Land Conflicts}
\begin{center}
\begin{adjustbox}{max height = .6\textheight}
\begin{tabular}{lcccc}
\hline \noalign{\smallskip} &  &  &  & ITT /\\
 &  & Control &  & control\\
Dependent Variable & N & mean & ITT & mean (\%)\\
 & (1) & (2) & (3) & (4)\\
\noalign{\smallskip}\hline \noalign{\smallskip}\textbf{Panel A: Ouctomes for all residents} &  &  &  & \\
 &  &  &  & \\
Any serious land dispute & 4,011 & 0.088 & -0.028 & -31.9\\
 &  &  & [0.015]* & \\
Any unresolved land dispute & 4,011 & 0.024 & -0.009 & -35.3\\
 &  &  & [0.006] & \\
Any threats, property damage, or violence & 4,011 & 0.041 & -0.014 & -34.3\\
 &  &  & [0.006]** & \\
\quad Property damage or violence in land dispute & 4,011 & 0.021 & -0.005 & -26.3\\
 &  &  & [0.004] & \\
\tab Threats & 4,011 & 0.035 & -0.015 & -41.8\\
 &  &  & [0.006]** & \\
\tab Property damage & 4,011 & 0.010 & -0.009 & -87.4\\
 &  &  & [0.005]* & \\
\tab Violence & 4,011 & 0.017 & -0.004 & -20.3\\
 &  &  & [0.004] & \\
\textbf{Panel B: Conditional on a land dispute} &  &  &  & \\
 &  &  &  & \\
Length of maximum land conflict & 353 & 12.844 & 5.362 & 41.7\\
 &  &  & [2.356]** & \\
Resolved land dispute & 353 & 0.662 & 0.039 & 5.9\\
 &  &  & [0.044] & \\
Any threats, property damage, or violence \phantom{} & 353 & 0.469 & -0.076 & -16.2\\
 &  &  & [0.044]* & \\
\quad Property damage or violence \tab & 353 & 0.238 & -0.042 & -17.7\\
 &  &  & [0.037] & \\
\tab Threats \phantom{} & 353 & 0.404 & -0.090 & -22.3\\
 &  &  & [0.042]** & \\
\tab Property damage \phantom{} & 353 & 0.114 & -0.092 & -81.2\\
 &  &  & [0.043]** & \\
\tab Violence \phantom{} & 353 & 0.198 & -0.015 & -7.6\\
 &  &  & [0.037] & \\
\tab Witchcraft \phantom{} & 353 & 0.070 & -0.056 & -80.8\\
 &  &  & [0.034]* & \\
\noalign{\smallskip}\hline\end{tabular}

\end{adjustbox}
\end{center}
\end{table}

\begin{table}[H]
\caption{Resident-Level Effect of years since Treatment on Land Conflicts: IV}
\begin{center}
\begin{adjustbox}{max width = \textwidth}
\begin{tabular}{lcccccccccc}
\hline \noalign{\smallskip} & \multicolumn{5}{c}{\uline{\hfill 1-year endline \hfill}} & \multicolumn{5}{c}{\uline{\hfill 3-year endline \hfill}}\\
Dependent Variable & Control mean & N & IV est. & SE & ITT / control mean & Control mean & N & IV est. & SE & ITT / control mean\\
 & (1) & (2) & (3) & (4) & (5) & (6) & (7) & (8) & (9) & (10)\\
\noalign{\smallskip}\hline \noalign{\smallskip}\textbf{Panel A: Ouctomes for all residents} &  &  &  &  &  &  &  &  &  & \\
Any serious land dispute & 0.222 & 1,900 & 0.011 & 0.027 & 4.8 & 0.085 & 1,692 & 0.018 & 0.013 & 20.9\\
Any unresolved land dispute & 0.067 & 1,900 & 0.008 & 0.011 & 11.5 & 0.023 & 1,692 & 0.004 & 0.006 & 18.8\\
Any threats, property damage, or violence & 0.121 & 1,900 & 0.017 & 0.019 & 14.3 & 0.038 & 1,692 & 0.007 & 0.006 & 18.2\\
\quad Property damage or violence in land dispute & 0.087 & 1,900 & 0.026 & 0.014* & 29.4 & 0.019 & 1,692 & 0.006 & 0.004* & 32.2\\
\tab Threats & 0.114 & 1,900 & 0.010 & 0.019 & 9.2 & 0.033 & 1,692 & 0.007 & 0.005 & 21.6\\
\tab Property damage & 0.039 & 1,900 & 0.010 & 0.009 & 26.6 & 0.009 & 1,692 & 0.004 & 0.002* & 47.0\\
\tab Violence & 0.075 & 1,900 & 0.016 & 0.013 & 21.4 & 0.016 & 1,692 & 0.005 & 0.004 & 32.7\\
\textbf{Panel B: Conditional on a land dispute} &  &  &  &  &  &  &  &  &  & \\
Length of maximum land conflict &  &  &  &  &  & 13.883 & 145 & -0.770 & 3.636 & -5.5\\
Resolved land dispute & 0.697 & 429 & -0.024 & 0.039 & -3.4 & 0.657 & 145 & 0.045 & 0.041 & 6.8\\
Any threats, property damage, or violence \phantom{} & 0.546 & 429 & 0.046 & 0.052 & 8.4 & 0.449 & 145 & 0.023 & 0.043 & 5.2\\
\quad Property damage or violence \tab & 0.393 & 429 & 0.092 & 0.042** & 23.5 & 0.225 & 145 & 0.037 & 0.033 & 16.5\\
\tab Threats \phantom{} & 0.511 & 429 & 0.018 & 0.050 & 3.5 & 0.384 & 145 & 0.030 & 0.044 & 7.9\\
\tab Property damage \phantom{} & 0.174 & 429 & 0.037 & 0.035 & 21.4 & 0.106 & 145 & 0.030 & 0.022 & 27.8\\
\tab Violence \phantom{} & 0.337 & 429 & 0.054 & 0.039 & 16.0 & 0.188 & 145 & 0.030 & 0.032 & 16.0\\
\tab Witchcraft \phantom{} &  &  &  &  &  & 0.062 & 145 & -0.006 & 0.019 & -9.6\\
\noalign{\smallskip}\hline\end{tabular}

\end{adjustbox}
\end{center}
\end{table}
\end{landscape}


\subsection{Effect of being dropped on violence outcomes}
\begin{table}[H]
\caption{Resident-Level Effect of being dropped on violence outcomes}
\label{conflict_adj_p_dropped}
\begin{center}
\begin{adjustbox}{max width = \textwidth}
\begin{tabular}{lcccccccccccc}
\hline \noalign{\smallskip} &  & \multicolumn{2}{c}{\uline{\hfill Control mean \hfill}} & \multicolumn{2}{c}{\uline{\hfill Effect of \hfill}} & \multicolumn{2}{c}{Effect as pct} &  &  &  &  & \\
 &  & Full sample & Dropped & Treatment w/ & Dropping & \multicolumn{2}{c}{\uline{\hfill of control \hfill}} &  &  &  &  & \\
Dependent variable & N & w/ E2 weights & comms. only & E2 weights & No weights & Treatment & Dropping &  &  &  &  & \\
 & (1) & (2) & (3) & (4) & (5) & (6) & (7) &  &  &  &  & \\
\noalign{\smallskip}\hline \noalign{\smallskip}\textbf{Panel A: Land dispute outomes for all residents} &  &  &  &  &  &  &  &  &  &  &  & \\
 &  &  &  &  &  &  &  &  &  &  &  & \\
Any serious land dispute & 3,995 & 0.22 & 0.21 & 0.009 & 0.002 &  & 4.23 & 0.86 &  &  &  & \\
 &  &  &  & (0.017) & (0.024) &  &  &  &  &  &  & \\
Any unresolved land dispute & 3,995 & 0.07 & 0.06 & -0.019 & -0.010 &  & -27.54 & -15.79 &  &  &  & \\
 &  &  &  & (0.008)** & (0.011) &  &  &  &  &  &  & \\
Any threats, property damage, or violence & 3,995 & 0.12 & 0.12 & -0.010 & -0.009 &  & -7.74 & -7.42 &  &  &  & \\
 &  &  &  & (0.013) & (0.016) &  &  &  &  &  &  & \\
\quad Property damage or violence in land dispute & 3,995 & 0.09 & 0.09 & -0.013 & -0.007 &  & -14.40 & -7.60 &  &  &  & \\
 &  &  &  & (0.010) & (0.014) &  &  &  &  &  &  & \\
\tab Threats & 3,995 & 0.11 & 0.12 & -0.005 & -0.005 &  & -4.48 & -4.68 &  &  &  & \\
 &  &  &  & (0.012) & (0.015) &  &  &  &  &  &  & \\
\tab Property damage & 3,995 & 0.04 & 0.03 & -0.017 & -0.024 &  & -38.74 & -85.25 &  &  &  & \\
 &  &  &  & (0.007)** & (0.010) &  &  &  &  &  &  & \\
\tab Violence & 3,995 & 0.08 & 0.08 & -0.009 & 0.000 &  & -10.96 & 0.16 &  &  &  & \\
 &  &  &  & (0.009) & (0.013) &  &  &  &  &  &  & \\
\textbf{Panel B: Conditional on a land dispute} &  &  &  &  &  &  &  &  &  &  &  & \\
 &  &  &  &  &  &  &  &  &  &  &  & \\
Resolved land dispute & 883 & 0.69 & 0.70 & 0.075 & 0.028 &  & 10.86 & 4.00 &  &  &  & \\
 &  &  &  & (0.029)** & (0.042) &  &  &  &  &  &  & \\
Any threats, property damage, or violence \phantom{} & 883 & 0.57 & 0.57 & -0.030 & -0.027 &  & -5.26 & -4.69 &  &  &  & \\
 &  &  &  & (0.039) & (0.049) &  &  &  &  &  &  & \\
\quad Property damage or violence \tab & 883 & 0.43 & 0.40 & -0.040 & -0.009 &  & -9.29 & -2.20 &  &  &  & \\
 &  &  &  & (0.033) & (0.047) &  &  &  &  &  &  & \\
\tab Threats \phantom{} & 883 & 0.53 & 0.54 & -0.021 & -0.017 &  & -3.90 & -3.10 &  &  &  & \\
 &  &  &  & (0.039) & (0.047) &  &  &  &  &  &  & \\
\tab Property damage \phantom{} & 883 & 0.20 & 0.13 & -0.062 & -0.086 &  & -30.63 & -64.29 &  &  &  & \\
 &  &  &  & (0.028)** & (0.041) &  &  &  &  &  &  & \\
\tab Violence \phantom{} & 883 & 0.37 & 0.35 & -0.029 & 0.013 &  & -7.74 & 3.61 &  &  &  & \\
 &  &  &  & (0.032) & (0.043) &  &  &  &  &  &  & \\
\textbf{Panel C: General dispute outcomes for all residents} &  &  &  &  &  &  &  &  &  &  &  & \\
 &  &  &  &  &  &  &  &  &  &  &  & \\
Any serious dispute & 3,995 & 0.30 & 0.28 & 0.024 & -0.022 &  & 8.11 & -7.65 &  &  &  & \\
 &  &  &  & (0.019) & (0.024) &  &  &  &  &  &  & \\
Any unresolved dispute & 3,995 & 0.12 & 0.10 & -0.010 & -0.019 &  & -8.64 & -18.48 &  &  &  & \\
 &  &  &  & (0.012) & (0.014) &  &  &  &  &  &  & \\
\textbf{Panel D: Conditional on a dispute} &  &  &  &  &  &  &  &  &  &  &  & \\
 &  &  &  &  &  &  &  &  &  &  &  & \\
Resolved dispute & 1,242 & 0.68 & 0.68 & 0.043 & 0.009 &  & 6.32 & 1.29 &  &  &  & \\
 &  &  &  & (0.025)* & (0.037) &  &  &  &  &  &  & \\
\quad Resolved via informal mechanism & 1,243 & 0.23 & 0.28 & 0.017 & 0.031 &  & 7.26 & 11.21 &  &  &  & \\
 &  &  &  & (0.027) & (0.041) &  &  &  &  &  &  & \\
\noalign{\smallskip}\hline\end{tabular}

\end{adjustbox}
\end{center}
\end{table}

\clearpage

\subsection{Community-Level Impacts}

\begin{table}[H]
\caption{Community-Level Effect of Treatment Assignment on Conflicts}
\begin{center}
\begin{adjustbox}{max width = \textwidth}
\begin{tabular}{lcccccccccccccc}
\hline \noalign{\smallskip} & \multicolumn{7}{c}{\uline{\hfill 1-year endline \hfill}} & \multicolumn{7}{c}{\uline{\hfill 3-year endline \hfill}}\\
 &  &  &  & ITT / &  &  &  &  &  &  & ITT / &  &  & \\
 & Control &  &  & control & Est. & WY Adj. & Holms Adj & Control &  &  & control & Est. & WY Adj. & Holms Adj\\
Dependent Variable & mean & N & ITT & mean (\%) & p-val & p-val & p-val & mean & N & ITT & mean (\%) & p-val & p-val & p-val\\
 & (1) & (2) & (3) & (4) & (5) & (6) & (7) & (8) & (9) & (10) & (11) & (12) & (13) & (14)\\
\noalign{\smallskip}\hline \noalign{\smallskip}Any Violence & 0.442 & 940 & 0.085 & 19.3 & 0.021 & 0.500\textsuperscript{a} & 0.193 & 0.622 & 971 & -0.057 & -9.1 & 0.319 & 1.000\textsuperscript{b} & 0.735\\
 &  &  & [0.037]** &  &  &  &  &  &  & [0.057] &  &  &  & \\
Level of community violence & 0.790 & 940 & 0.152 & 19.2 & 0.070 & 1.000\textsuperscript{a} & 0.478 & 0.984 & 971 & -0.131 & -13.3 & 0.171 & 1.000\textsuperscript{b} & 0.677\\
 &  &  & [0.083]* &  &  &  &  &  &  & [0.096] &  &  &  & \\
\quad Intertribal violence & 0.028 & 940 & 0.008 & 30.0 & 0.512 & 1.000\textsuperscript{a} & 0.884 & 0.021 & 971 & -0.016 & -77.8 & 0.039 & 0.500\textsuperscript{b} & 0.331\\
 &  &  & [0.013] &  &  &  &  &  &  & [0.008]** &  &  &  & \\
\quad Violent strike or protest & 0.061 & 940 & -0.004 & -7.1 & 0.782 & 1.000\textsuperscript{a} & 0.952 & 0.002 & 971 & 0.015 & 912.4 & 0.104 & 0.500\textsuperscript{b} & 0.585\\
 &  &  & [0.016] &  &  &  &  &  &  & [0.009] &  &  &  & \\
\quad Youth-elder dispute & 0.110 & 940 & 0.044 & 40.2 & 0.124 & 1.000\textsuperscript{a} & 0.603 & 0.103 & 967 & 0.003 & 3.4 & 0.880 & 1.000\textsuperscript{b} & 0.880\\
 &  &  & [0.029] &  &  &  &  &  &  & [0.023] &  &  &  & \\
\quad Peaceful strike or protest & 0.100 & 940 & 0.037 & 37.1 & 0.144 & 1.000\textsuperscript{a} & 0.606 & 0.059 & 971 & 0.010 & 16.0 & 0.613 & 1.000\textsuperscript{b} & 0.850\\
 &  &  & [0.025] &  &  &  &  &  &  & [0.019] &  &  &  & \\
\quad Interfamily land disputes & 0.274 & 940 & 0.030 & 11.1 & 0.408 & 1.000\textsuperscript{a} & 0.878 & 0.548 & 971 & -0.071 & -12.9 & 0.282 & 1.000\textsuperscript{b} & 0.735\\
 &  &  & [0.037] &  &  &  &  &  &  & [0.066] &  &  &  & \\
\quad Conflicts with other towns & 0.154 & 940 & -0.005 & -3.5 & 0.870 & 1.000\textsuperscript{a} & 0.952 & 0.171 & 970 & -0.038 & -22.2 & 0.194 & 1.000\textsuperscript{b} & 0.677\\
 &  &  & [0.033] &  &  &  &  &  &  & [0.029] &  &  &  & \\
\quad Witch hunts & 0.015 & 940 & 0.023 & 153.3 & 0.087 & 1.000\textsuperscript{a} & 0.517 & 0.011 & 971 & -0.008 & -71.8 & 0.085 & 0.500\textsuperscript{b} & 0.551\\
 &  &  & [0.013]* &  &  &  &  &  &  & [0.005]* &  &  &  & \\
\quad Trial by ordeal & 0.048 & 940 & 0.019 & 39.4 & 0.298 & 1.000\textsuperscript{a} & 0.830 & 0.070 & 971 & -0.027 & -39.1 & 0.120 & 1.000\textsuperscript{b} & 0.592\\
 &  &  & [0.018] &  &  &  &  &  &  & [0.018] &  &  &  & \\
\noalign{\smallskip}\hline\end{tabular}

\end{adjustbox}
\end{center}
\end{table}


\clearpage
\subsection{Plot-Level Impacts}


\begin{table}[H]
\caption{Plot-Level Effect of Treatment Assignment on Investments}
\begin{center}
\begin{adjustbox}{max width = \textwidth}
\begin{tabular}{lcccc}
\hline \noalign{\smallskip} &  &  &  & ITT /\\
 &  & Control &  & control\\
Dependent Variable & N & mean & ITT & mean (\%)\\
 & (1) & (2) & (3) & (4)\\
\noalign{\smallskip}\hline \noalign{\smallskip}Property investment index, z-score & 7,861 & 0.018 & -0.052 & -287.8\\
 &  &  & [0.026]* & \\
\quad Monetary value of improvement, house and farm, (z-score) & 7,848 & 0.016 & -0.044 & -279.1\\
 &  &  & [0.028]** & \\
\quad Non-monetary improvement, house and farm (z-score) & 7,848 & 0.007 & -0.018 & -252.9\\
 &  &  & [0.025] & \\
\quad Made an improvement, house and farm & 7,861 & 0.360 & -0.028 & -7.8\\
 &  &  & [0.012]*** & \\
\noalign{\smallskip}\hline\end{tabular}

\end{adjustbox}
\end{center}
\end{table}


\begin{table}[H]
\caption{Plot-Level Effect of Treatment Assignment on Security}
\begin{center}
\begin{adjustbox}{max width = \textwidth}
\begin{tabular}{lcccc}
\hline \noalign{\smallskip} &  &  &  & ITT /\\
 &  & Control &  & control\\
Dependent Variable & N & mean & ITT & mean (\%)\\
 & (1) & (2) & (3) & (4)\\
\noalign{\smallskip}\hline \noalign{\smallskip}Security index through rights & 7,516 & 0.041 & -0.078 & -191.2\\
 &  &  & [0.030]*** & \\
\quad Feels secure in boundaries, house and farm & 8,022 & 0.873 & -0.025 & -2.9\\
 &  &  & [0.008]*** & \\
\quad Has ability to inherit, house and farm & 8,022 & 0.813 & -0.048 & -6.0\\
 &  &  & [0.010]*** & \\
\quad Has ability to sell, house and farm & 8,022 & 0.268 & -0.010 & -3.6\\
 &  &  & [0.013] & \\
\quad Has ability to pawn, house and farm & 8,022 & 0.264 & -0.012 & -4.5\\
 &  &  & [0.013] & \\
\quad has ability to survey, house and farm & 8,022 & 0.619 & -0.016 & -2.5\\
 &  &  & [0.013] & \\
\noalign{\smallskip}\hline\end{tabular}

\end{adjustbox}
\end{center}
\end{table}

\begin{table}[H]
\caption{Plot-Level Effect of Treatment Assignment on Main Outcomes}
\begin{center}
\begin{adjustbox}{max width = \textwidth}
\begin{tabular}{lcccc}
\hline \noalign{\smallskip}Dependent Variable & N & Control Mean & ATE & ATE / control mean\\
 & (1) & (2) & (3) & (4)\\
\noalign{\smallskip}\hline \noalign{\smallskip}Security rights index, aggregated & 4,011 & 0.045 & -0.083 & -183.5\\
 &  &  & [0.037]** & \\
Improvement index, aggregated & 4,011 & 0.023 & -0.065 & -275.5\\
 &  &  & [0.037]* & \\
Index of Fallow Land, farm & 3,666 & 0.003 & -0.003 & -109.3\\
 &  &  & [0.044] & \\
Size of farm & 3,598 & 37.481 & 2.750 & 7.3\\
 &  &  & [1.051]*** & \\
\noalign{\smallskip}\hline\end{tabular}

\end{adjustbox}
\end{center}
\end{table}


\end{document}